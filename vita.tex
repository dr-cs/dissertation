\chapter*{Vita}
\addcontentsline{toc}{chapter}{Vita}  %add Vita section to Table of Contents


Chris Simpkins is the oldest of four brothers. An Air Force brat born in Biloxi, Mississippi at Keesler AFB, he lived in Germany from ages 6 through 13. He speaks German and French and enjoys returning to Europe to teach in study abroad programs and see old friends. After his dad retired from the Air Force Chris moved with his family to Shady Spring, West Virginia.

Chris began his professional life in the U.S. Air Force where he was a pilot, software engineer, and instructor. He was selected as the first ever first-assignment space instructor upon graduation from Undergraduate Space Training (UST) in 1992, where he taught basic Newtonian mechanics. He returned to flying after three years as a classroom instructor and interactive courseware developer to fly the KC-135, rising to the highest co-pilot position (Stan/Eval), being one of only two co-pilots in his squadron to earn special-ops qualification, and winning an outstanding crew award while deployed in Saudi Arabia. He finished his Air Force career doing one of the things he love most: teaching, this time as a T-37 instructor pilot at Columnbus AFB, MS. He left the Air Force in 2000 to pursue his other professional passion: computing.

Chris is currently a Lecturer in Computer Science at the Georgia Institute of Technology. He completed his MS in Computer Science in 2004 specializing in Artificial Intelligence. A 1990 graduate of the United States Air Force Academy, his background includes research, software engineering, flying, and teaching. During 15 years as a professional software engineer in private industry, the military, and applied research, he built and delivered dozens of successful enterprise-scale and single-user systems, mostly as chief architect or lead software engineer. As a researcher, he has applied machine learning to text analysis, radio emmiter identification, automated antenna design, and adaptive agent technology.
%% His doctoral thesis is focused on enabling the engineering of scalable intelligent agent software. To this end he is creating an experimental programming language that represents a new adaptive agent paradigm. AFABL (A Friendly Adaptive Behavior Language) integrates reinforcement learning into a Scala-based domain-specific language which makes certain kinds of agent programs less complex to write and understand.
