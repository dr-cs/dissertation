\documentclass[12pt]{gatech-thesis}
\usepackage{amsmath,amssymb,latexsym,float,epsfig,subfigure,graphicx,listings,hyperref}

% \includeonly{aase}

\title{Integrating Reinforcement Learning into a Programming Language}

\author{Christopher Simpkins}
\department{School of Interactive Computing}


\principaladvisor{Professor Charles L. Isbell, Jr.}[School of
  Interactive Computing]
\committeechair{}
\firstreader{Dr. Douglas Bodner}[Tennenbaum Institute]
\secondreader{Professor Mark Riedl}[School of Interactive Computing]
\thirdreader{Dr. Spencer Rugaber}[School of Computer Science]
\fourthreader{Professor Andrea Thomaz}[School of Interactive Computing]


\degree{Doctor of Philosophy}


%% Set \listmajortrue below, then uncomment and set this for
%% interdisciplinary PhD programs so that the title page says
%% ``[degree] in [major]'' and puts the department at the bottom of
%% the page, rather than saying ``[degree] in the [department]''

%% \major{Algorithms, Combinatorics, and Optimization}
\major{Computer Science}

\copyrightyear{2016}
\submitdate{15 December 2016}

%% The date the last committee member signs the thesis form. Printed
%% on the approval page.
\approveddate{TBD}


\bibfiles{references}

%% The following are the defaults
%%    \titlepagetrue
%%    \signaturepagetrue
%%    \copyrightfalse
%%    \figurespagetrue
%%    \tablespagetrue
%%    \contentspagetrue
%%    \dedicationheadingfalse
%%    \bibpagetrue
%%    \thesisproposalfalse
%%    \strictmarginstrue
%%    \dissertationfalse
%%    \listmajorfalse
%%    \multivolumefalse

\listmajortrue
\dissertationtrue

%% \thesisproposaltrue

\begin{document}
\nocite{*}
\bibliographystyle{gatech-thesis}
%%
\begin{preliminary}
\begin{dedication}
\null\vfil
{\large
\begin{center}
To my children,\\\vspace{12pt}
Isaac and Meredith,\\\vspace{12pt}
who motivated my decision to give up flying and enter academia.
\end{center}}
\vfil\null
\end{dedication}
%\begin{preface}

%\end{preface}
%\begin{acknowledgements}

%\end{acknowledgements}
% print table of contents, figures and tables here.
\contents
% if you need a "List of Symbols or Abbreviations" look into
% gatech-thesis-gloss.sty.

\begin{summary}
\vspace{-1in}

{\bf My Thesis:} Improving the composability of modules in modular reinforcement learning (MRL) and integrating MRL into a programming language supports reuse in adaptive agent software engineering.  This thesis claims that (1) MRL can be extended to support composability in modular reinforcement learning agents by uncoupling the reward scales of the modules that comprise the agents, and (2) integrating modular reinforcement learning into a programming language supports adaptive agent programming by reducing the effort required to adapt agent programs to new domains.

Composability, an essential property of modularity in software engineering, allows components to be reused in the construction of new systems.  In the case of a modular reinforcement learning agent composability means being able to reuse behavior modules in new agents without modifying the modules.  The current state of the art in modular reinforcement learning supports decomposition but not composition, or module reuse.  This dissertation contributes a reformulation of MRL that supports composition by uncoupling the reward scales of the modules that comprise a modular reinforcement learning agent, and an algorithm that solves it.

An excellent way to support software engineering is with practical, usable programming languages.  The second major contribution of this dissertation is an domain-specific language and framework embedded in the Scala programming language that integrates MRL.  We show how this integration is useful to software engineers writing practical adaptive agent software by applying AFABL to representative agent programming tasks and measuring the benefit of integrated MRL compared to traditional programming.  This application to practical software engineering problems distinguishes AFABL from previous work in integrating RL into programming languages such as ALisp.

\end{summary}

\end{preliminary}

\cleardoublepage

\chapter{Introduction}

This chapter sets the stage for the work presented in this dissertation, an overview of its contributions, and concludes with a roadmap of the following chapters.

\section{The Promise of AI and the Challenges of Software Engineering}

Artificial intelligence was one of the first grand promises of computing. Almost as soon as the field of computer science was born the pioneers of AI were promising machines that could think and act like humans within the ``visible future.'' Early research in AI focused on machines that could ``think'' like humans and employed symbolic computation to create constraint solvers, planners, and exert systems that used truth maintenance systems. We had computer programs that solved algebra equations, found paths, played games, and diagnosed illnesses based on user-reported symptoms. But symbolic, or knowlege-based AI hit a bottleneck in the late 1980s -- commonly called the knowledge acquisition bottleneck -- and the early promise of developing systems that {\it replaced} humans faded. But AI in general did not fade. AI reinvented itself. Instead of creating systems of rules and inferening engines based on encoded knowledge, modern AI applies well-developed models from mathematics and engineering to problems traditionally considered part of AI. Text retrieval uses the vector space model, speech recognition uses Hidden Markov Models, ...

Software engineering has struggled to keep pace with the growing size and complexity of the systems. Over time the field of software engineering, both in academia and industry, has developed a well-defined set of practices and design guidelines that result in software systems that are maintainable, reliable and extensible. Programming languages have been the primary means by which research in software engineering and formal computer science has been brought to bear for the working programmer. From structured programming to object-oriented programming to powerful modern type systems, important advances in computing research have real impact when they are incorporated as features in practical programming languages. In the same way that, say, formal methods are used by the modern programmer in the form of static type systems without requiring the programmer to know much about formal methods, AFABL's goal is to allow the programmer to use reinforcement learning without knowing much about reinforcement learning algorithms.

One can think of reinforcement learning (RL) as a machine learning approach to planning, that is, a way of finding a sequence of actions that achieves a goal.  The RL problem formulation is this: an agent's world is described by a set of states, the agent can execute one of a set of actions in each state, and the agent is rewarded to greater or lesser degrees for each state-changing action it executes. For the software engineer who would like to employ refinforcement learning without becoming an expert in reinforcement learning the most important thing to understand is that the world of an agent can be modeled in terms of states, actions, and rewards. Our work shows that if you can specify the states, actions, and rewards for an agent our algorithms can work behind the scenes to develop a control policy.


\section{Contributions}

The work work presented in this dissertation marries articficial intelligence and software engineering in a way that advances both fields. The needs of practical software engineering for reuse and composability inspires a new AI algorithm for modular reinforcement learning. Integrating this new formulation of modular reinforcement learning and associated algorithms into a programming language enables a new kind of software engineering: modular adaptive agent programming. In particular, this work makes the following contributions:

\begin{itemize}
\item We explain a problem with the current state of the art in modular reinforcement learning, namely, that performance degrades if subagents have differing, incomparable reward scales.
\item We empirically demonstrate the performance degradation of modular reinforcement learning agents whose subagent have incomparable reward scales.
\item We present an analysis of the shortcoming of current approaches to modular reinforcement learning based on Arrow's Impossibility Theorem for social choice in order to frame our solution.
\item We reformulate the modular reinformcement learning problem as one of {\it command arbitration} instead of merging MDPs or Q functions.
\item We present a command arbitration algirithm -- Arbi-Q -- that uses our theoretically grounded reformulation of modular reinforcement learning.
\item We empirically demonstrate that modular reinforcenment learning agents using Arbi-Q exhibit no performace degradation when subagents have incomparable reward scales.
\item We present a Scala-embedded domain-specific langage -- AFABL -- that integrates modular reinformcent learning and our Arbi-Q command arbitration algorithm.
\item We demonstrate and quantify the value of integrating integrating modular reinformcenet learning into a programming language to practical software engineering in a programmer study applying AFABL in a syntheic agent programming domain.
\item We apply AFABL to a practical problems in psychology-based human agent modeling to partially demonstrate AFABL's practical potential.
\end{itemize}

\section{Outline}

Chapter \ref{ch:rl} provides background information in modular reinformcement learning and existing approaches that sets the stage for the rest of the dissertation.

Chapter \ref{ch:arbiq} presents an empircal demonstration of the performance degradation of modular reinforcement learning agents whose subagent have incomparable reward scales. Arrow's Impssibility Theorem for social choice provides an explanation for the failure of existing approaches to modular reinformcent learning and a framework for our solution. We present our solution, the Arbi-Q command arbitration algorithm, and empirically demonstrate that it does not exhibit the same performance degradation as existing approaches to modular reinformcement learning.

Chapter \ref{ch:se} provides background information on software engineering that motivates the use of modular reinformcement learning in building practical software systems, and the integration of modular reinforcement learning into a programming language.

Chapter \ref{ch:afabl} presents a programming language, AFABL, which integrates modular reinformcement learning. AFABL, a domain-specific language embedded in the Scala language, allows programmers to write adaptive software agents in a declarative style using elements of modular reinforcement learning: subagents (modules) with states, actions, and rewards. We present the results of a programmer study that shows the value of integrating reinforcement learning into a programming language: AFABL agents are less complex, easier to write, and easier to adapt to changes in the environment.

Chapter \ref{ch:applications} presents a practical application of AFABL that further demonstrates the usefulness of integrating modular reinforcement learning into a programming language.

Chapter \ref{ch:conclusion} concludes the dissertation by reviewing how the central theses of this dissertation were confirmed and the present work's context, limitations, and consequences. We relate modular reinforcement learning to the broader field of decompositional reinformcement learning and discuss directions for future work.


\chapter{Background in Reinforcement Learning}\label{ch:rl}

The field of reinformcement learning is large and growing. In this chapter we provide the background in reinforcement learning and modular reinformcenet learning necessary to understand our work and place it in context.

\section{Reinforcement Learning}

One can think of reinforcement learning (RL) \cite{sutton1998reinforcement,kaelbling1996reinforcement} as a machine learning approach to planning, that is, as a way of finding a sequence of actions that achieves a goal. In RL, problems of decision-making by agents interacting with uncertain environments are usually modeled as Markov decision processes (MDPs). In the MDP framework, at each time step the agent senses the state of the environment and executes an action from the set of actions available to it in that state. The agent's action (and perhaps other uncontrolled external events) cause a stochastic change in the state of the environment. The agent receives a (possibly zero) scalar reward from the environment. The agent's goal is to find a {\it policy}; that is, to choose actions so as to maximize the expected sum of rewards over some time horizon. An optimal policy is a mapping from states to actions that maximizes the long-term expected reward.  In short, a policy defines which action an agent should take in a given state to maximize its chances of reaching a goal.

\subsection{Markov Decision Processes}

The basic Markov decision process is a 3-tuple,

\[
(S, A, T(s, a, s'))
\]

where

\begin{itemize}
\item $S$ is a set of states,
\item $A$ is a set of actions, and
\item $T(s, a, s')$ is a transition function which gives the probably that executing action $a$ in state $s$ will result in $s'$.
\end{itemize}

Some authors include a reward function, $R(s, a, s')$ which specifies the reward received by an agent for taking action $a$ in state $s$ and arriving in state $s'$, or equivalently simply a reward for arriving in state $s$, $R(s)$. Still others include an initialization function, $I(s)$, which specifies the probablity the the agent will start in some state $s \in S$, and others even include a discount factor, $\gamma$, which specifies how much longer-term rewards are discounted compared to shorter-term rewards.

We prefer to think of the basic 3-tuple MDP as representing the states and state transition dynamics of a world, and an agent solving a Markov Decision {\it Problem} which adds the reward function, initialization function, and discount factor. This distinction between Markov Decision {\it Processes} and Markov Decision {\it Problems} leads to a more natural expression of worlds and agents for programmer who are not familiar with the underlyiung theory of reinforcement learning. For example, most people would not include a single universal reward function as part of the ``world'' because different agents may value different states differently. Similarly, different agents may have a shorter or longer term view of decision optimality and thus different discount factors. The solution to a Markov Decision Problem is a policy, denoted $\pi$, which is a mapping from states to actions.


\section{The Bunny World}

Value functions: how good is it for an agent to be particular state. The value of a state (or state, action pair) is computed from the rewards and state transition dynamics of a world.

The value of a state under a particular policy $\pi$ is given by:

\[
V^\pi(s) = E_\pi\{R_t | s_t = s\} = E_\pi \{ \sum_{k=0}^{\infty} \gamma^k r_{t+k+i} | s_t = s \}
\]




\subsection{Solving MDPs with Dynamic Programming}

\subsubsection{Policy Iteration}

\subsubsection{Value Iteration}

\subsection{Learning Policies via Reinforcement Learning}

\subsubsection{Markov-Chain Monte-Carlo Policy Learning}

\subsubsection{Q-Learning}

\subsubsection{SARSA}


\subsection{The Curse of Dimensionality in Reinforcement Learning}


\section{Decompositional Reinforcement Learning}



\subsection{Temporal Decomposition: Hierarchical Reinformcement Learning}

Semi-markov decision processes.

Current implementations of partial programming are based on hierarchical reinforcement learning (HRL) \cite{dietterich1998maxq,dietterich2000hierarchical,sutton1999between,parr1998reinforcement,andre2000programmable,andre2002state,marthi2005concurrent}, which exploits a temporal decomposition of the Q function.  In the case of HRL, the designer typically specifies a delegation hierarchy of components with points of adaptation where a policy is learned to perform the delegation.  The designer programs the policies of some of these components, which constitute a partial specification of the agent's behavior, and some of the components use reinforcement learning to adapt to the hierarchy by learning the control policies for their parts of the problem.  The adaptive components relieve the designer from writing the parts of the program that are hard to specify, or require difficult to write adaptivity, and the partial program constrains the learning problem faced by the adaptive components, which speeds convergence.  Components can be reused in other contexts, providing for modularity in temporal problem decompositions.

\subsubsection{Semi-Markov Decision Processes}

\subsubsection{MAXQ}

\subsubsection{Options}

\subsubsection{Hierarchical Abstract Machines}


\subsection{Concurrent Decomposition: Modular Reinforcement Learning}

A second kind of decomposition in reinforcement learning, which is somewhat confusingly referred to as modular reinforcement learning (MRL) ~\cite{russell2003q-decomposition,sprague2003multiple-goal}, decomposes the original problem concurrently rather than temporally. In contrast to HRL, current MRL implementations do not involve delegation; instead, an agent is decomposed into several components, each of which is concurrently learning a subgoal of the original, complex, multiple-goal learning problem.

Two approaches to modular reinforcement learning: merging MDPs \cite{singh1998how-to-dynamically}, and merging Q functions \cite{sprague2003multiple-goal,russell2003q-decomposition}.

On each occasion that a decision for what action to take is needed, an arbitrator combines the action preferences of the components to compute the output of the joint policy.  The current state of the art in modular reinforcement learning uses the the Q-values of the components directly to effect the arbitration decision.  Russell's and Zimdars's Q-decomposition \cite{russell2003q-decomposition} views the problem as a value function decomposition.  Sprague and Ballard view the problem more explicitly as arbitration, that is, action selection among multiple concurrently executing modules, although their solution, Greatest-Mass (GM) Sarsa \cite{sprague2003multiple-goal} is equivalent to Q-decomposition.


\chapter{Command Arbitration for Modular Reinforcement Learning}\label{ch:arbiq}

In this chapter we demonstrate the performance degradation of modular reinforcement learning agents whose module have incomparable reward scales. Arrow's Impssibility Theorem for social choice provides an explanation for the failure of existing approaches to modular reinformcent learning and a framework for our solution. We present our solution, the Arbi-Q command arbitration algorithm, and empirically demonstrate that it does not exhibit the same performance degradation as existing approaches to modular reinformcement learning.

\section{Modular reinforcement learning}

In the previous chapter we explained the state of the art in modular reinforcement learning as a decomposition of an implicitly global Q-function in to additive modules. To support software engineering we would like these components to be truly modular. In particular, we would like the components to be reusable and easily understood by agent authors.  Unfortunately, current approaches implicitly require a global, monolithic reward signal, which detracts from these properties.  In particular, it detracts from the ability of the agent author to locally define the reward for a component because the reward scales of other modules must be taken into account.

\subsection{Command Arbitration}

Like any software engineering, agent programming would benefit from modularity.  Truly modular reinforcement learning would facilitate speed of learning convergence, state abstraction, and transfer -- a module written in one context can be reused in another context because the component is dependent only on its own local reward signal.  In this section we discuss the difficulties that current approaches face in achieving true modularity and present a new formulation which solves the core problem in Section \ref{sec:mrl-solution}.

To properly investigate our proposed mechanism of combining local learners, we will restrict our attention to the MRL framework. In this framework, a learning agent, $M$, is represented by a set of $n$ subproblems (modules), $M=\{M_j\}_{j=1}^n$, each having its own reward signal and state space. The typical formulation is to have subproblems share an action set, thus $M_j = (S_j,A,R_j)$. When the agent takes an action in the world, each subproblem's state is updated, and each subproblem receives a reward. The agent can be thought of as having a state space that is a cross product of the subproblem state spaces, $S=S_1\times S_2\times\ldots\times S_n$. Equivalently, each module can be considered to have a state space that is a subset of the global state space -- a module state that is an abstraction of the world state. Traditionally, it is assumed that the agent's reward is a sum of subproblem rewards, $R(s,a)=\sum_{j\in N} R_j(s,a)$, where $N=\{1,\ldots,n\}$. Thus, $M$ is well-defined as a reinforcement learning problem, $M=(S,A,R)$, and the goal is to find a joint policy $\pi(s)$, composed of locally learned policies $\pi_j(s)$, that is optimal for the whole agent. We let $Q_j(s,a)$ refer to the Q-value learned by component $j$ for state $s$ and action $a$. Again, traditionally, $Q(s,a)=\sum_{j\in N} Q_j(s,a)$.

\subsection{Merging local signals}

Difficulty arises when multiple single goals are being combined in a larger, multi-goal learning problem. Take AvoidWolf and FindFood for instance; it is fairly straightforward to code an internally consistent reward signal for each.  However, it is unclear how to combine the two into the larger task of LiveLong.  For example, if there is no penalty for failing to eat within a certain time period (starvation), then the obvious policy is to avoid the predator and ignore the food.  Such a degenerate case could also happen if the reward signal for one learning module were scaled without re-engineering the other reward signals in the system, a problem explained by Bhat, et., al. \cite{bhat2006on-the-difficulty}. For example, one could imagine swapping-in a new AvoidWolf module with a reward several times higher than the previous module, such that avoiding the predator always carries higher reward than finding the food.  In this case, a delegating agent would favor AvoidWolf to the near exclusion of FindFood.  The ability to substitute learning modules without modifying the rest of the system is one of the primary benefits of true modularity, and this modularity is difficult to achieve if local reward signals must be merged.

\subsection{Ideal Arbitration is Impossible}

Aside from the practical challenges cited above, Bhat, et., al.  \cite{bhat2006on-the-difficulty} showed that ideal arbitration is impossible in full generality because command arbitration reduces to Arrow's Paradox \cite{arrow1963social}: an agent is a ``society'' of modules, and command arbitration is social choice.  The problem is that we want the arbitrator to have the following reasonable properties:

\begin{itemize}

\item \textbf{Universality}: the ability to handle any possible set of
  modules.

\item \textbf{Unanimity}: guarantee that if every module prefers
  action A, so will the arbitrator.

\item \textbf{Independence of Irrelevant Alternatives}: each
  module's preference for actions A and B are independent of the
  availability of any other action C, prevents any particular
  module from affecting the global action choice by dishonestly
  reporting its own preference ordering.

\item \textbf{Scale Invariance}: ability to scale any module's
  Q-values without affecting the arbitrator's choice.  This is the
  crucial property that allows separately authored modules with
  incomparable reward signals.

\item \textbf{Non-Dictatorship}: no module gets its way all the time.

\end{itemize}

According to Arrow's Paradox, if $|A|\geq 3$, then there does not exist an arbitration function that satisfies each of the properties listed above.  So even simple agents with more than three actions are too complex for theoretically ideal arbitration.  This dissertation contributes a novel formulation of MRL and an algorithm that implements it, which we discuss in Section \ref{sec:mrl-arbiq}.

\section{Reformulating MRL}

Bhat, et., al., \cite{bhat2006on-the-difficulty} argued for a ``benevolent dictator'' arbitrator function but left the arbitration algorithm unspecified.  The contribution of this work is to show that a first-cut arbitration algorithm embodying the ideas in \cite{bhat2006on-the-difficulty} performs competitively with other MRL algorithms and shows superior robustness to module modification.  This robustness to module modification is the chief enabler of truly modular reinforcement learning in which modules can be transferred from one system to another without having to re-engineer the reward signals to fit the new host system.

In this work, we relax the dictatorship requirement in order to make a practical arbitrator-based learning algorithm feasible.  In the next section we present our Arbi-Q algorithm based on this practicalized arbitrator-based framework.

%% This is a ``fix'' because it is a reformulation. It doesn't get around
%% the impossibility result but instead defines a different thing to
%% optimize.  This fits though...now all learners are more homogenous;
%% they all have a local reward signal to optimize.

%% So while it is hard to write once-and-for-all a reward signal for a
%% multiple-goal learning problem, it is easier to write a collection of
%% single-goal reward signals, and hopefully combine them in a meaningful
%% way. Whereas in previous work, we show that there is no
%% general-purpose arbitration rule for combining subgoal preferences, in
%% the present work we make the hypothesis that the arbitrator itself is
%% a subgoal and thus has it's own notion of what it means to do its
%% subgoal well, i.e, has it's own reward signal. This means, given the
%% appropriate state, the supergoal can learn to select substeps
%% optimally. The question is what state constitutes ``appropriate
%% state''. The uninteresting case is the one in which the supergoal
%% state is simply the world state, in which case we have garnered no
%% saving, as this is simply flat RL all over again.

%In discussing MRL techniques, we will follow the nomenclature as
%introduced by ~\cite{sprague03multiple}.

\subsection{Formalization}

Our reformulation of MRL adds a {\em command arbitrator} \cite{brooks1986a-robust}.  The arbitrator has a state space that may be the same as or different from the modules' state spaces, an action set that represents choosing a module, $A_{CA} = {1 ... n}$, and a reward signal that represents the ``greater good.'' The arbitrtator's reward function, $R_a(s)$, is independently defined rather than being derived from the module rewards.  It is now another part of the problem specification; in the partial programming setting, this corresponds to $R_a(s)$ being human-authored. Note that $R_a(s)$ may or may not be equal to the sum of the rewards of the agent's modules. In fact, the module rewards $R_i(s,a)$ may not have any correlation with the arbitrator's reward $R_a(s)$.

The agent's policy is defined indirectly by the arbitrator's policy, $\pi_{CA}(s, a)$, which assigns probabilities to the selection of each module's preferred action for each state.  This formulation relaxes the non-dictatorship requirement of ideal arbitration if you think of the arbitrator as a special module.  By Arrow's theorem, other properties will still hold.

For the agent author, this formulation adds the requirement of authoring a dedicated reward signal for the arbitrator.  For our bunny agent, this is LiveLongProsper:

\begin{itemize}
\item {\em Why} avoid predator, {\em why} eat? To live longer.
\item Encodes the tradeoffs between modules -- perhaps food is more
  important to some bunnies.
\item The arbitrator could be hand-authored, or could be another RL
  agent.
\end{itemize}

For the small cost of authoring a reward signal that represents the ``greater good'' you get true modularity, that is, the ability to combine separately authored modules with incomparable rewards.  This new reward signal is now the metric we use to measure the performance of the agent.

In our MRL framework an agent is an arbitrator plus a list of modules. Formally, an agent consists of the following elements:

\begin{enumerate}
\item A reward function for the command arbitrator, $R_{CA}(s)$,
\item An action set $A$ for the agent as a whole, shared by each module,
\item A set of reinforcement learning modules, $M$
\item A state abstraction function, $moduleState_i$ for each module $m_i$
\end{enumerate}

In the next section we present a reinforcement learning-based command arbitration algorithm.

\section{The Arbi-Q Command Arbitration Algorithm}\label{sec:mrl-arbiq}

Our reformulation of MRL is based on an independently specified arbitrator \cite{brooks1986a-robust} that is itself a reinforcement learner. The state space for the arbitrator is the world state -- no state abstraction is used for the arbitrator. The arbitrator's action set, $A_{CA}$, is a set of integer indexes to the agent's list of modules (we use $CA$ in subscript to refer to the command arbitrator and numbers or $i$ to refer to modules).  As with any reinforcement learner, the arbitrator learns a policy. In the case of the arbitrator this policy, $\pi_{CA}$ is a mapping from states to modules. The modules' policies are mappings from abstracted module state to actions in the world, that is, the agent's actions. The policy defines which module chooses the agent's action in a particular state.

Arbi-Q uses thje Sarsa Q-Learning algorithm to learn the arbitrator's policy. At each time step the arbitrator uses its policy to select a module, then the module uses its local policy to select an action that the agent executes. The results of executing the action are communicated to the arbitrator as a consequence of the module selection, and to the modules as a consequence of action selection. Each module uses a state abstracion and function to transform the world state into the the subset of the state relevant to the module, and a reward function that is based on the module's state abstraction. In this way the modules are coupled to the world in which they operate -- the modules can only operate in worlds which contain the state features expected by its state abstraction function -- but the modules are not coupled to other modules or to an arbitrator. The Arbi-Q algorithm is detailed in Algorithm \ref{alg:arbiq}


\begin{algorithm}
  \caption{Arbi-Q}\label{alg:arbiq}
  \begin{algorithmic}
    \State $Q_{CA} \gets$ random initial values
    \For{each module $i$}
       \State $Q_{i} \gets$ random initial values
    \EndFor
    \For{each episode}
      \State $s \gets$ world.initialState()
      \State $m \gets \epsilon-$greedy action for $s$ from $\pi_{CA}$ derived from $Q_{CA}$ \Comment{choose module}
      \State $s_{m} \gets moduleState(s)$ \Comment{abstract state for module}
      \State $a \gets \epsilon-$greedy action for $s_{m}$ from $\pi_m$ derived from $Q_m$
      \Repeat
        \State Execute $a$, observe effects $r_{CA}$ and $s'$
        \State $a' \gets \epsilon-$greedy action for $s'$ from $\pi$ derived from $Q$
        \State $Q_{CA}(s, a) \gets Q_{CA}(s, a) + \alpha [R_{CA}(s) + \gamma Q_{CA}(s', a') - Q_{CA}(s, a)]$
        \For{each module $i$}
          \State $s'_{i} \gets moduleState(s')$ \Comment{abstract state for module}
          \State $r_i \gets reward(s_i)$ \Comment{module-specific reward}
          \State $Q_i(s, a) \gets Q_i(s, a) + \alpha [r_i + \gamma Q(s_i', a') - Q(s_i, a)]$
        \EndFor
        \State $s \gets s'$
        \State $a \gets a'$
      \Until $s$ is terminal
    \EndFor
  \end{algorithmic}
\end{algorithm}

Notice that Arbi-Q is learning its command arbitration policy at the same time the modules are learning their subtasks. While it is possible in principle to train the modules before training the arbitrator, emprical results show faster convergence when the arbitrator and modules are trained at the same time.

\section{Experiments}

Our principal claim is that Arbi-Q is robust to modules with incomparable reward scales, which would be an authoring error in existing MRL approaches. Our experiments show that GM-Q/Q-decomposition degrades when a module is modified to have an incomparable reward scales and that Arbi-Q is robust to such modification.

\subsection{Bunny-Wolf World}

We use a world derived from Sprague and Ballard ~\cite{sprague2003multiple-goal}.  In Bunny-Wolf world, our agent is a bunny that must eat food and avoid being eaten by a wolf.  The bunny world is a continuing world rather than an episodic world. There is no specified start state and there is no termination of episides. When the bunny finds and eats food, a new food item appears elsewhere. When the wolf eats the bunny the bunny ``respawns'' in a new location, similar to video games. We can represent such a bunny agent in our formulation as follows:

\begin{itemize}

\item Module 1: FindFood.  The bunny agent must find food in order to continue living.  When the bunny finds food it gets a reward of 1.0. In each step that it does not eat the bunny gets a reward of -0.1 to represent building hunger.

\item Module 2: AvoidWolf.  The bunny agent must avoid the wolf. Meeting the wolf gives the bunny a reward of -1.0. In each time step that the bunny avoids the wolf the bunny receives a reward of 0.1.

\item Agent's overall goal (implemented in arbitrator): LiveLongProsper.  For a given agent, being happy can mean dying young with a full stomach, or dying old but less sated.  Either conception of happiness requires the FindFood and AvoidWolf modules defined below.  The arbitrator, in a sense, encodes the agent's ``value system'' by representing the relative weight the agent assigns to its modules. To facilitate comparison to GM-Sarsa, LiveLongProsper's goal is to get as much food per time step as possible, which will require balancing food finding with wolf avoidance. This performance metric is essentially a composite of the comparably scaled modules of the GM-Sarsa agent: 0.0 for meeting the wolf, 1.0 for finding food, and 0.5 for each step in which the wolf is avoided but no food is eaten.

\end{itemize}


We validate Arbi-Q's performance by comparing it with Greatest Mass Sarsa, which is a Q-decomposition algorithm that orders actions by their summed Q-value, $X_a=\sum_j Q_j(s,a)$. We evaluate each algorithm similarly to Sprague and Ballard \cite{sprague2003multiple-goal} by training for 100,000 time steps and evaluating performance every 10,000 steps.  Performance is evaluated by running the greedy policy in the world for 1000 episodes and calculating the average reward per episode.  Each algorithm used a discount rate of 0.9 and $\epsilon$-greedy action selection during training with $\epsilon$ linearly discounted from 0.4, as in Sprague and Ballard's experiments.

For baselines, GM and Arbi-Q algorithms used modules with similarly scaled rewards. For robustness validation, we scaled the AvoidWolf module reward by 10 to simulate the swapping out of separately-authored learning modules.  We believe that a truly modular arbitrator function should handle such module modification without serious degradation of performance.  Otherwise, any time a learning module were modified, the arbitrator, and possibly all the other modules, would need to be modified to ensure compatibility.

\section{Results}

Emprical results clearly show that the performance of GM-Sarsa degrades when the reward scales of the modules are not comparable. The learning curves depicted in Figure \ref{fig:gmsarsa-results} show that GM-Sarsa bunny agent with incomparable reward scales for its modules converges to a lower score than with comparable rewards. To see why this is the case, consider a simplified example of the composite Q-values computed by GM-Sarsa with comparable rewards (B is for bunny, F is for food, W is for wolf):

\begin{center}
\begin{tabular}{|p{1em}|p{1em}|p{1em}|p{1em}|p{1em}|}\hline
  &   & B &   & F \\\hline
  &   &   & W &   \\\hline
  &   &   &   &   \\\hline
  &   &   &   &   \\\hline
  &   &   &   &   \\\hline
\end{tabular}
\end{center}

With comparable rewards the Q-value of moving right and left for FindFood would be

\begin{align*}
Q^*(s, a) &= \gamma \sum_{s'} T(s, a, s') \max_{a'} Q(s', a')
\end{align*}

With comparable rewards the Q-value of moving right and left for AvoidWolf would be

\begin{align*}
Q^*(s, a) &= \gamma \sum_{s'} T(s, a, s') \max_{a'} Q(s', a')
\end{align*}

And the composite Q-value of moving right would be

\begin{align*}
Q^*(s, a) &= \gamma \sum_{s'} T(s, a, s') \max_{a'} Q(s', a')
\end{align*}.

And the composite

If we scale the FindFood by 10, the Q-values for moving right and left would be

\begin{align*}
Q^*(s, a) &= \gamma \sum_{s'} T(s, a, s') \max_{a'} Q(s', a')
\end{align*}.

And the composite Q-values would be

\begin{align*}
Q^*(s, a) &= \gamma \sum_{s'} T(s, a, s') \max_{a'} Q(s', a')
\end{align*}.

As you can see, scaling the FindFood module dominates the decision problem, resulting in the bunny getting eaten, which gives negative reward and loses the food.


\begin{figure}[ht]
  \begin{center}
    \scalebox{.75}{\includegraphics{gm-bunny-wolf.png}}
    \caption{Performance of GM-Sarsa/Q-decomposition on the bunny-wolf problem. The learning curves show that Greatest Mass command arbitration degrades significantly when its module rewards are incomprable.}
  \end{center}
  \label{fig:gmsarsa-results}
\end{figure}


Arbi-Q does not exibit this problem because it does not use the Q-values of its modules directly. Instead, Arbi-Q learns when it should listen to a particular module. More precicely, Arbi-Q develops a probablity distribution for each state which says which module has the best advice. So Arbi-Q learns, for example, that when the wolf is close AvoidWolf should decide the bunny agent's action, and when the wolf is comfortably distant FindFood should decide the bunny agent's action.

\begin{figure}[ht]
  \begin{center}
    \scalebox{.75}{\includegraphics{arbiq-bunny-wolf.png}}
    \caption{Performance of Arbi-Q on the bunny-wolf problem. Arbi-Q converges to similar scores as GM-Sarsa and shows no degredation in performance when modules have incomprable rewards, suggesting that it is amenable to ``swappable'' modules.}
  \end{center}
  \label{fig:arbiq--results}
\end{figure}


For the comparable cases of both


\subsection{Related Work}

Rohanimanesh and Mahadevan extended the options HRL framework to concurrent settings in which multiple agents executing multiple simultaneous actions \cite{rohanimanesh2001decision,rohanimanesh2002learning}. Their work differs from ours in that their framework applies to a single agent taking multiple actions or multiple agents taking simultaneous actions, whereas we are concerned with a single agent executing a single action that is decided by multiple reinforcement learning modules.

Marthi and colleagues \cite{marthi2005concurrent} suggest extending their work in concurrent ALisp to include the Q-decomposition algorithm of Russel and Zimdars \cite{russell2003q-decomposition}, but this line of research was not pursued. Lau and colleagues developed a modular reinforcement learning system that uses a central coordinator for multiple concurrent MPDs \cite{lau2012coordination}. Lau's work differs form ours in that they develop a constraint system in the central coordinator that limits the allowable actions of the component reinforcement learners, thereby constraining their learning. Our approach does not require the arbitrator to know details of component learners, and component learners require no explicit or implicit knowledge of the arbitrator or the other components.

Due to the curse of dimensionality, abstraction of various kinds has long been an active area of research in reinforcement learning. One thread in abstraction is to use examples to guide abstraction. Zang and colleagues used examples of nearly optimal action sequences, or trajectories, to dynamically discover options from data, delivering speedups of up to 30 times in some cases \cite{zang2009discovering}. Learning from demonstration \cite{zang2010batch} uses human input to improve reinforcement learning performance. Zang and colleagues developed a value function approximation algorithm that leveraged human input to speed convergence for function approximation-based reinforcement learning algorithms \cite{zang2010using}. Cobo Rus and colleagues' Abstraction from Demonstration technique uses human demonstrations to infer state abstractions and builds policies based on those state abstractions \cite{cobo-rus2011automatic,cobo-rus2012automatic,cobo-rus2014abstraction}.

Another thread in abstraction seeks to use models from the physical world to create abstracions of (simulated) physical state spaces. Cobo Rus and colleagues created abstractions of state spaces by organizing the state spaces into classes of objects and using non-optimal Q-functions to estimate the risk of ignoring certain classes of objects. Cobos Rus's Object-Focused Q-Learning (OFQ) acheived exponential speedups in some cases \cite{cobo-rus2013object}. Scholz and colleagues developed Physics-Based Reinforcement Learning \cite{scholz2014physics}, which uses computational physics engines succh as Box2D \cite{catto2013box2d} as model representations, resulting in more sample-efficient learning compared to traditional object-oriented MDP approaches. Physics-based reinforcement learning was then applied successfully in robotic mobile manipulation \cite{scholz2015learning} and robot navigation \cite{scholz2016navigating} applications.

\section{Conclusion}


\chapter{Background in Software Engineering}\label{ch:se}

Software engineering is the process of creating software that is correct, reliable, and maintainable. This chapter provides background in software engineering that relates to our work. In particular, we claim in the next chapter that AFABL provides two benefits that have been central issues in software engineering: AFABL facilitates reuse and reduces complexity. Here we discuss the issue of reuse in software engineering and relate reuse to domain-specific languages, then briefly discuss complexity measurement in software engineering. Finally, we close with a discussion of adaptive programming.

\section{Software Reuse}

Software reuse was identified as a primary tool in improving software engineering practice since the birth of the field of software engineering in 1968 \cite{mcilroy1968mass}. Software reuse means using an existing software artifact in a new software system, ideally without modifying the original artifact \cite{krueger1992a-software,frakes2005a-software}. Reusable artifacts may be source code libraries, components, programming languages, and application frameworks \cite{polancic2010a-an-empirical} as well as concepts such as software schemas, architectures, and design patterns. The benefits of software reuse seem obvious, and empirical studies have indeed shown that reuse reduces defect rates, reduces refactoring costs, and increases productivity \cite{basili1996a-how-reuse,mohagheghi2008a-an-empirical}. In the next chapter we also quantify the reduction in code complexity afforded by the AFABL DSL. DSLs, as we discuss below, are a particular kind of reusable artifact.

Krueger presents a useful framework for understanding and assessing reuse techniques. Of particular interest to our work, he discusses reuse techniques in terms of cognitive distance, which he defines as an intuitive measure of the effort required to use a reusable software artifact in the process of turning the concept of a software application into a working system. The smaller the cognitive distance between a reusable software artifact and the concept of the application program in which it is to be reused, the more successful the reuse. Thus, abstraction is crucial to software reuse, the higher the level of abstraction the better. Krueger proposes three techniques for minimizing the cognitive distance in reusable software artifacts: ``(1) using fixed and variable abstractions that are both succinct and expressive, (2) maximizing the hidden parts of the abstractions, and (3) using automated mappings from abstraction specifications to abstraction realizations'' \cite{krueger1992a-software}. Krueger was the first to recognize that high level languages such as C and Java are themselves examples of software reuse -- language constructs are abstraction specifications, assembly language or byte code are abstraction realizations. DSLs (or VHLLs -- Very High Level Languages -- as he called them) are also examples of software reuse that raise the level of abstraction even higher, offering abstraction specifications that are entities in some problem domain such as set theory or circuit design. In the next chapter we will analyze AFABL according to Krueger's framework.

Gacek argues for creating domain-specific reference architectures to facilitate reuse\cite{gacek1995a-exploiting}. A domain-specific application architecture identifies all of the components that comprise a software application for a particular domain and the interactions between the components. A domain-specific language can be seen as a domain-specific architecture, whose components are modeled as language abstractions. AFABL, in this sense, is a domain-specific architecture for agents with multiple continuing goals.

\subsection{Domain-Specific Languages}

A domain-specific language (DSL) is a language that provides constructs and semantics tailored to a specific problem domain. Specialized programming languages were already widespread when Landin proposed the first unified framework for designing domain specific languages in 1966 \cite{landin1966next}. In Landin's framework the design of a DSL consists of two independent parts: the written form of the language, and the kinds of abstractions that can be expressed in the language. Every language has an abstract syntax, axiomatization, and an underlying abstract machine. Today many DSLs are in widespread use for various application domains such as typesetting and manuscript preparation (\TeX and \LaTeX), circuit design (VHDL), web page authoring (HTML and CSS), data exchange (JSON and XML), and many more.

Perhaps the most successful DSL, with which every reasonably literate software engineer or computer scientist is familiar, is Structured Query Language (SQL) \cite{}. SQL was originally presented as SEQUEL in 1974 by Chamberlin and Boyce of IBM Research \cite{chamberlin1974sequel}. Today SQL is used in all significant relational databases and its ANSI/ISO standard is on its third version. SQL has succeeded so completely because it provides exactly the right abstractions and semantics for using relational databases.

Domain-specific languages provide two primary benefits in software engineering: improving programmer productivity and improving communication with domain experts \cite{fowler2011domain}. In Chapter \ref{ch:afabl} we show that AFABL improves programmer productivity by reducing the effort required to write agents and reducing the complexity of agent code. In Chapter \ref{ch:applications} we present an application of AFABL to the domain of personality modeling in psychology to demonstrate AFABL's usefulness as a tool for bridging specialist knowledge from a non-computing domain with computational models.

Hudak argued that a domain-specific language is the ``ultimate abstraction,'' providing abstractions and semantics tailored to a particular application domain, but that languages are typically difficult to implement and difficult to evolve as the domain is better understood and changes need to be made to the language \cite{hudak1996building}. To solve this problem Hudak argued for and demonstrated domain-specific {\it embedded} languages, that is, DSLs embedded in a general-purpose host language, inheriting the tooling, syntax, and semantics of the host language while adding domain-specific constructs. Hudak used Haskell and showed how higher-order typed languages were particularly well-suited for hosting DSELs \cite{hudak1998modular}. For AFABL we used Scala, which has similarities to Haskell, as we discuss in the next chapter.

Ward \cite{ward1994language} proposed Language Oriented Programming as a way of organizing software development. Instead of developing reusable libraries in a general purpose language the software engineer develops a formal domain-specific specification of the application, then implements this specification as a DSL. He calls the resulting software development process ``middle-out'' development, where the DSL is created first (the middle layer) then the DSL is implemented (the lower layer) and specific applications are developed using the DSL (the upper layer). He presents several examples of this approach, including \LaTeX \cite{lamport1986document} as a collection of \TeX \cite{knuth1984texbook} macros and Emacs \cite{stallman2014emacs}, which is essentially a Lisp interpreter with addressable memory buffers -- what users think of as the editor is actually implemented as a collection of Emacs Lisp functions, making Emacs infinitely extensible.
Neighbors was the first to propose {\it domain languages} specifically for the purpose of reuse \cite{neighbors1984draco}. Lorenz and colleagues tie together the concept of reuse with Language Oriented Programming in software engineering \cite{lorenz2011a-code}, comparing the effort and benefits of implementing and using internal versus external DSLs. They performed a case study in which they created an external DSL (using a language workbench system called Meta-Programming System MPS \cite{dmitriev2004language}) and internal DSL using the experimental LOP language Cedalion \cite{rosenan2010designing} for the same task (calculator software product line). They found that both approaches achieved their code reuse goals but, while both DSLs took similar effort to use, the external DSL took four times longer to implement. They concluded that internal DSLs should be favored over external DSLs for most software development projects. Rosenan argues that host languages themselves can be designed with internal DSL creation in mind -- a kind of ``Language-Oriented Programming language'' -- and presents a LOP language as a proof of concept \cite{rosenan2010designing}. As we discuss in the next chapter, while not explicitly advertised as an LOP language, hosting DSLs has been a design goal of the Scala language that we use to host AFABL.


%% \cite{taha2008domain-specific}

%% \cite{dmitriev2004a-language}


%% \cite{mitchell1993on-abstraction}

%% \cite{simpkins2008towards}


%% \cite{zang2007towards}


%% \cite{mernik2005when}


\section{Software Complexity}

A second goal of AFABL is reducing the complexity of agent code. For our purposes we define complexity as a measure of the effort required to understand, modify, or test a piece of code. Many kinds of complexity measures have been proposed, including function point analysis \cite{albrecht1979measuring}, Halstead's ``software science'' \cite{halstead1977elements}, information flow \cite{henry1981software} and many others. But perhaps the most widespread complexity measure is McCabe's cyclomatic complexity \cite{mccabe1976complexity,mccabe1989design}. Roughly speaking, the McCabe cyclomatic complexity number is a measure of the number of unique paths through a program.

Though there are critiques of cyclomatic complexity
\cite{gill1991cyclomatic} and proposed modifications \cite{weyuker1988evaluating}, McCabe's cyclomatic complexity measure is still in widespread use and has been shown to be a useful and valid measure. Curtis and colleagues found that both Halstead and McCabe complexity metrics correlate with psychological complexity of code, especially for less experienced programmers \cite{curtis1979measuring}. Finally, McCabe's cyclomatic complexity number has a simple method of calculation which makes it particularly appealing. We explain McCabe's cyclomatic complexity in the next chapter.

\section{Adaptive Programming}

By adaptive software we refer to the notion used in the machine learning community: software that learns to adapt to its environment during run-time, not software that is written to be easily changed by modifying the source code and recompiling.  In particular, we use Peter Norvig's definition of adaptive software:

\begin{quote}
Adaptive software uses available information about changes in its
environment to improve its behavior~\cite{norvig1998adaptive}.
\end{quote}

In this work we are particularly interested in programming intelligent agents that operate in real environments, and in virtual environments that are designed to simulate real environments.  Examples of these kinds of agents include robots, and non-player characters in interactive games and dramas.  Unlike traditional programs, agents operate in environments that are often incompletely perceived and constantly changing.  This incompleteness of perception and dynamism in the environment creates a strong need for adaptivity.  Programming this adaptivity by hand in a language that does not provide built-in support for adaptivity is very cumbersome.  Due to its integration of reinforcement learning, AFABL provides this kind of adaptivity, making the construction of adaptive agents much easier.


\subsection{How to Achieve Adaptive Software}

Norvig identifies several requirements of adaptive soft\-ware---adaptive programming concerns, agent-oriented concerns, and software engineering concerns---and five key technologies---dynamic programming languages, agent technology, decision theory, reinforcement learning, and probabilistic networks---needed to realize adaptive software.  These requirements and technologies are embodied in his model of adaptive programming given in Table~\ref{tab:adaptive-model}.

\begin{table}[h]

\begin{center}
\begin{tabular}{|c|c|}\hline
Traditional Programming & Adaptive Programming \\ \hline
Function/Class & Agent/Module \\
Input/Output & Perception/Action \\
Logic-based & Probability-based \\
Goal-based & Utility-based \\
Sequential, single- & Parallel, multi- \\
Hand-programmed & Trained (Learning) \\
Fidelity to designer & Perform well in environment \\
Pass test suite & Scientific method\\ \hline
\end{tabular}
\caption{Peter Norvig's model of adaptive programming
  ~\cite{norvig1998decision}.}
\label{tab:adaptive-model}
\end{center}

\end{table}

AFABL integrates two of Norvig's key technologies: agent technology and reinforcement learning.

%% This dissertation will explain how AFABL implements Norvig's adaptive programming model and argue that AFABL satisfies many of Norvig's requirements.

\subsection{The Partial Programming Paradigm}

The model of computation, or ``control regime,'' supported by a language is the fundamental semantics of language constructs that molds the way programmers think about programs. PROLOG provides a declarative semantics in which programmers express objects and constraints, and pose queries for which PROLOG can find proofs.  In C, programmers manipulate a complex state machine. Functional languages such as ML and Haskell are based on Lambda Calculus. AFABL, being a domain-specific language (DSL) \cite{hudak1996building} embedded in Scala \cite{odersky2008programming,odersky2005scalable}, is effectively multi-paradigmatic, supporting functional and object-oriented programming through its direct use of Scala, and partial programming semantics based on reinforcement learning, in which the programmer defines the agent's actions and allows the learning system to select them based on states and rewards.  Thus partial programming represents a new paradigm which results in a new way of writing programs that is much better suited to certain classes of problems, namely adaptive agents, than other programming paradigms.  AFABL facilitates adaptive agent programming in the same way that PROLOG facilitates logic programming.  While it is possible to write logic programs in a procedural language, it is much more natural and efficient to write logic programs in PROLOG.  The issue here is not Turing-completeness, the issue is cognitive load on the programmer.  In a Turing-complete language, writing a program for any decidable problem is theoretically possible, but is often practically impossible for certain classes of problems.  If this were not true then the whole enterprise of language design would have reached its end years ago.

The essential characteristic of partial programming that makes it the right paradigm for adaptive software is that it enables the separation of the ``what'' of agent behavior from the ``how'' in those cases where the ``how'' is either unknown or simply too cumbersome or difficult to write explicitly.  Returning to our PROLOG analogy, PROLOG programmers define elements of logical arguments.  The PROLOG system handles unification and backtracking search automatically, relieving the programmer from the need to think of such details. Similarly, in AFABL the programmer defines elements of behaviors -- states, actions, and rewards -- and leaves the language's runtime system to handle the details of how particular combinations of these elements determine the agent's behavior in a given state.  AFABL allows an agent programmer to think at a higher level of abstraction, ignoring details that are not relevant to defining an agent's behavior.  When writing an agent in AFABL, the primary task of the programmer is to define the actions that an agent can take, define whatever conditions are known to invoke certain behaviors, and define other behaviors as ``adaptive,'' that is, to be learned by the AFABL's integrated reinforcement learning.  This ability to program partial behaviors relieves a great deal of burden from the programmer and greatly simplifies the task of writing adaptive agents.  In the next chapter we will see how AFABL implements its support for adaptivity and partial programming.

\subsection{Related Work in Adaptive Programming}

%% \subsubsection{ALisp}

There is already a body of work in integrating reinforcement learning into programming languages, mostly from Stuart Russell and his group at UC Berkeley ~\cite{andre2001programmable,andre2002state}.  Their work is based on {\it hierarchical reinforcement learning}~\cite{parr1998reinforcement,dietterich1998maxq}, which enables the use of prior knowledge by constraining the learning process with hierarchies of partially specified machines.  This formulation of reinforcement learning allows a programmer to specify parts of an agent's behavior that are known and understood already while allowing the learning system to learn the remaining parts in a way that is consistent with what the programmer specified explicitly.

The notion of {\em programmable hierarchical abstract machines} (PHAM) ~\cite{andre2001programmable} was integrated into a programming language in the form of a set of Lisp macros (ALisp) ~\cite{andre2002state}. Andre and Russell provided provably convergent learning algorithms for partially specified learning problems and demonstrated the expressiveness of their languages, paving the way for the development of RL-based adaptive programming. Our work builds on theirs except that AFABL integrates modular, rather than hierarchical reinforcement learning, and we validate the software engineering benefits through a programmer study.

%% \subsubsection{Adaptation-Based Programming}

Bauer's Ph.D. work in adaptation-based programming \cite{bauer2013adaptation} is the closest to ours in its focus on the practical application of adaptive programming. Bauer implemented libraries for automated adaptation as a Java library \cite{bauer2011adaptation} and as a Haskell embedded DSL \cite{bauer2011adaptation-haskell}. These systems used Q-learning internally but did not use modular reinforcement learning. Bauer also did not conduct empirical software engineering studies of programmers to quantify and qualify the benefits of integrating reinforcement learning into a programming language. In the next chapter we present our language, AFABL, which integrates modular reinforcement learning and report the results of a programmer study that demonstrates its value.


\chapter{AFABL: A Friendly Adaptive Behavior Language}\label{ch:afabl}

AFABL is an internal domain-specific language shallowly embedded in the Scala programming language. In this chapter we explain why we chose to implement AFABL as a Scala-embedded DSL, present the basic elements of AFABL with examples, and report the results of a programmer study which confirm and quantify the usefulness of integrating reinforcement learning into a programming language.

\section{Why an embedded DSL?}

We chose to implement AFABL as a shallowly embedded domain-specific language because of the exploratory nature of this research. Our goal at this point is to confirm our expectation that integrating reinforcement learning is useful to programmers writing agents, to explore the nature of this integration, and to get qualitative feedback from programmers on their experience using a language that integrates reinforcement learning. Writing a full language with its own lexical and syntactic structure, internal representations, and language translators and tools (interpreters, compilers, linkers, etc.) would distract from the core questions we are trying to answer. As we discuss in Section \ref{sec:conclusion-full-language}, creating a full independent language is a direction for future research whcih will be guided by the results we present here, for which an embedded DSL is sufficient.

\section{Why Scala?}

Hosting DSLs is a primary design goal of the Scala programming language. Scala is an expressive and concise language which already enables the expression of domain models with little syntactic baggage. By employing just a few Scala language features that are designed for writing expressive and convenient libraries, we can create a DSL that Scala programmers will find familiar and non-Scala programmers can use to encode adaptive agents. Indeed, the nature of Scala's syntax and language features is such that many Scala libraries that are not explicity labeled as DSLs qualify as shallow embedded DSLs.


\section{AFABL Concepts}

AFABL is a language for encoding adaptive (intelligent) agents. Before we present the AFABL language, we review basic adaptive agent concepts that can be encoded in AFABL.

\subsection{Terminology}

\begin{itemize}

\item State

  A state is a configuration of all objects and agents in a world. These configurations can include object locations and orientations, mental or emotional dispositions of agents, or indications of the occurrence of events (e.g., agent X just got shot).

\item Perception, or Percept

  A subset of a world state that is perceived by an agent (or module).

\item Action

  An action is a one-shot state manipulation that can be executed by an agent.  Sometimes called "primitives," actions acquire greater meaning when they are executed (possibly in sequences) by behavior modules to achieve a goal or satisfy a constraint.


\item World

  Sometimes called an "environment," a world is a container for all the things an agent can perceive and act upon, and possibly hidden things.  Worlds are represented by states that specify the configuration of all the things in a world at a given point in time.

\item World Dynamics

  The dynamics of a world specifies the state transitions that are made in response to the execution of actions.  World dynamics may be deterministic or stochastic.  Markov Decision Processes are often used to represent world dynamics.

\item Behavior Module

  A behavior module is a self-contained agent component that recommends an action (response) for a given state perception (stimulus).  Anything that produces a policy (a mapping from states to actions and tasks) is a behavior module.  A behavior module is sometimes called a "subagent" in modular reinforcement learning.

\item Agent

  An agent is an entity that acts under its own control, perceiving the state of its world and executing actions in response to these perceived states.

\item Intelligent Agent

  An intelligent agent is an agent that chooses its actions to achieve goals or satisfy constraints.

\item Adaptive Agent

  An adaptive agent is an intelligent agent that can automatically adapt its behaviors to different worlds (that is, choose different actions sequences to achieve goals in worlds with different dynamics), or adapt its behaviors at run-time to worlds in which the dynamics change.

\item Modular Agent

  A modular agent is an agent that consists of multiple behavior modules and performs command arbitration to decide which module's preferred action to execute in a given state.

\item Command Arbitration

  Command arbitration is the act of deciding, for a given state, which


\end{itemize}

\subsection{Agent Architecture}

An AFABL agent is a behavorial agent that is composed of reusable behavior modules.  By "behavioral agent" we mean that the agent executes an action in resopnse to a stimulus, represented by a state observation.  Each behavior module is itself an agent that has a preferred action for each state.  AFABL agents perform command arbitration to choose one of the modules' recommended actions for each state.  The behavior modules recommend actions in each state, and the arbitrator chooses which module to "listen to" in each state.  The AFABL agent architecture is a subsumption architecture \cite{brooks1986a-robust}.

\subsubsection{Behavior Modules}

Behavior modules, sometimes called subagents in the modular reinforcement learning iterature, are agents that are meant to be combined to form larger agents.  Behavior modules are similar to the layers of Brooks's subsumption architecture with an important difference: autonomy.  The internal working of a behavior module is never altered externally.  A behavior module defines a state abstraction that converts the state observation it is given to a (possibly) simpler state that is used internally for decision making and learning.  The decision making and adaptation mechanisms inside a module remain completely under the module's control.  Interaction with the module consists entirely of reporting a state observation to the module, asking the module for an action, and reporting to the module the effect of executing an action.

\subsubsection{Adaptive Modules}

An adaptive module employs learning algorithms under the hood to achieve automatic adaptivity.  By adaptive we mean two things: (1) adaptation to new worlds, and (2) run-time adaptaion.  A module that is programmed to work for worlds with a given state representation will work with any world that provides the same (or greater) state representation, even if the dynamics of the worlds differ.  An adaptive module need simply be retrained for the new world.  Once an adaptive module is running in an active agent, the module may continue to tune its internal learning models as the agent acts in the world, providing for run-time adaptation.

\subsubsection{Command Arbitrators}

Command arbitrators take as input the action preferences of a set of modules, and selects one of the actions.  If the command arbitrator is part of a module, the action selected is the preferred action of the module.  If the command arbitrator is part of the top-level agent, then the actoin selected is the action that will be executed by the agent.

\section{The AFABL Language}

AFABL is a framework for implementing adaptive agents in the Scala programming language.

\subsection{Worlds}

Every AFABL module and agent is designed to operate in a world. A world defines the states and state transition dynamics for a given state type and action types. Details are discussed below.

\subsubsection{States}

The states of a world can be represented with any kind of Scala class. Case classes (\ref{sec:scala-case-classes}) are good for representing states because of their concise syntax and built-in equality methods. Figure \ref{fig:bunny-state-code} shows a case class for a state with three state variables: the locations of a bunny, wolf, and food.

\begin{figure}[h]
\begin{center}

\begin{lstlisting}[language=Scala]
case class Location(x: Int, y: Int)

case class BunnyState(
  bunny: Location,
  wolf: Location,
  food: Location
)
\end{lstlisting}

\caption{Scala code to represent states in the bunny world.}
\end{center}
\label{fig:bunny-state-code}
\end{figure}

\subsubsection{Actions}

Actions are represented by objects which can be instances of any class. As with states, case classes make a good choice. Figure \ref{fig:bunny-action-code} shows actions for the bunny world implemented as a Scala enumeration (\ref{sec:scala-enumerations}).

\begin{figure}[h]
\begin{center}

\begin{lstlisting}[language=Scala]
object BunnyAction extends Enumeration {
  val Up = Value("^")
  val Down = Value("v")
  val Left = Value("<")
  val Right = Value(">")
}
\end{lstlisting}

\caption{Scala code to represent the actions that the bunny agent can take in the bunny world.}
\end{center}
\label{fig:bunny-action-code}
\end{figure}

\subsubsection{World Dynamics}

An agent executes actions in a world, and those actions potentially change the state of the world. Having defined Scala representations for states and actions, we can define a world. Figure \ref{fig:world-code} shows the abstract class which defines the basic interface of world objects. As we discuss in Sections \ref{sec:afabl-modules} and \ref{sec:afabl-agents}, all modules and agents are defined to act in a particular instance of a world. As with states and actions, world representations make no advanced use of the Scala programming language.

\begin{figure}[h]
\begin{center}

\begin{lstlisting}[language=Scala]
/** A stateful object representing world dynamics.
  */
abstract class World[S, A] {

  /** Initialize the world to a start state.
    */
  def init(): S

  /** Move agent to a start state, according to the rules of the
    * implementing world. Useful for continuing worlds where there is
    * no terminal state and the agent "respawns" after dying.
    */
  def resetAgent(): S

  /** All the states of this world. Necessary for reinforcement learning
    * agents.
    */
  def states: Seq[S]

  /** All the actions an agent can execute in this world.  For
    * simplicity this includes all the actions, and the actions that
    * aren't available in a given state simply have no
    * effect. Necessary for reinforcement learning agents.
    */
  def actions: Seq[A]

  /** Execute action in the world, resulting in a (possibly) new state.
   */
  def act(action: A): S
}
\end{lstlisting}
\caption{The abstract superclass of all world classes for AFABL agents.}
\end{center}
\label{fig:world-code}
\end{figure}

Figure \ref{fig:bunny-world-code}

\begin{figure}[h]
\begin{center}

\begin{lstlisting}[language=Scala]
class BunnyWorld(val width: Int = 5, val height: Int = 5)
  extends World[BunnyState, BunnyAction.Value] with LazyLogging {

  // Initialize the world state
  var state = init()

  // In Scala, defs can be overriden with vals
  val states = {
    // Calculate every possible combination of locations for the
    // bunny, wolf, and food
  }

  // This line returns all the values of the BunnyAction enumeration
  val actions = BunnyAction.values.toSeq

  def init(): BunnyState = {
    // Caclulate initial locations for the bunny, wolf, and food.
    //
  }

  def resetAgent(): BunnyState = {
    // "Respawn" the bunny at a new location, update the world state
    // and return the new state
  }

  def act(intendedAction: BunnyAction.Value): BunnyState = {
    // Code to calculate the actual action due to uncertainty in the
    // environment and update the state of the world based on the
    // actual action.
  }
  // Helper functions ...
}
\end{lstlisting}

\caption{Parts of the bunny world class showing important aspects of the implementation of the {\tt World} class.}
\end{center}
\label{fig:bunny-world-code}
\end{figure}


\subsection{Modules}

Figure \ref{fig:find-food-code} shows a possible AFABL implementation of a behavior module that represents the goal of finding food. First is the definition of a case class, {\tt FindFoodState}, to represent the state abstraction for FindFood modules. {\tt FindFoodState} includes only two of the three state variables in the bunny world.

\begin{center}
\begin{lstlisting}[language=Scala,frame=none]
case class FindFoodState(bunny: Location, food: Location)
\end{lstlisting}
\end{center}

We store a reference to an {\tt AfablModule} for FindFood in {\tt findFood}.

\begin{center}
\begin{lstlisting}[language=Scala,frame=none]
val findFood = AfablModule(
\end{lstlisting}
\end{center}

The {\tt AfablModule} factory method takes three comma-separated arguments: an instance of a {\tt World} that the module can act and learn in, a {\tt stateAbstraction} function, and a {\tt moduleReward} function.

The first argument to {\tt AfablModule} is the world:

\begin{center}
\begin{lstlisting}[language=Scala,frame=none]
world = new BunnyWorld
\end{lstlisting}
\end{center}

The {\tt world} and {\tt =} are optional, but if included must be verbatim, i.e., considered part of the AFABL language.

A state abstraction function that a world state object parameter and returns an instance of our state abstraction class:

\begin{center}
\begin{lstlisting}[language=Scala,frame=none]
stateAbstraction = (worldState: BunnyState) => {
  FindFoodState(worldState.bunny, worldState.food)
}
\end{lstlisting}
\end{center}

The {\tt stateAbstraction} and {\tt =} are optional, but if included should be considered part of the AFABL language. {\tt worldState} is a user-chosen name, {\tt BunnyState} must match the state type defined for the world in which the module and agent operate, in this case it's the first type parameter to {\tt World} in the {\tt BunnyWorld} code in Figure \ref{fig:bunny-world-code}. The last expression in the body of the {\tt stateAbstraction} function must be an instance of a module state, in this case {\tt FindFoodState}.

A module reward function takes an instance of our state abstraction class and returns the reward this module receives for begin in that state:

\begin{center}
\begin{lstlisting}[language=Scala,frame=none]
moduleReward = (moduleState: FindFoodState) => {
  if (moduleState.bunny == moduleState.food) 1.0
  else -0.1
}
\end{lstlisting}
\end{center}

The {\tt moduleReward} and {\tt =} are optional, but if included should be considered part of the AFABL language. {\tt moduleState} is a user-chosen name, but the parameter type, {\tt FindFoodState} in this example, must match the return type of the {\tt stateAbstraction} function. The last expression in the body of the {\tt moduleReward} function must be a {\tt Double} value. In this case, which is typical, the body of the {\tt moduleReward} function is an {\tt if} expression which simply returns the reward based on state predicates. This example is another case where we could have implemented DSL-specific syntax, such as a list of predicates and values, but the syntactic overhead of Scala's {\tt if} expression is minimal and the code is crystal clear to any Scala programmer.

These three components -- world, state abstraction and module reward -- define a module specific learning problem on a subset of the world in which the module (and agent containing the module) may act. Internally, AFABL instantiates a Sarsa learning algorithm that uses these components using the algorithm parameters discussed in Chapter \ref{ch:arbiq}, but the programmer need not be aware of any details of reinforcement learning {\it algorithms}. The AFABL programmer need only be familiar with the reinforcement learning {\it problem}.

\begin{figure}[h]
\begin{center}

\begin{lstlisting}[language=Scala]
case class FindFoodState(bunny: Location, food: Location)

val findFood = AfablModule(
  world = new BunnyWorld,

  stateAbstraction = (worldState: BunnyState) => {
    FindFoodState(worldState.bunny, worldState.food)
  },

  moduleReward = (moduleState: FindFoodState) => {
    if (moduleState.bunny == moduleState.food) 1.0
    else -0.1
  }
)
\end{lstlisting}

\caption{AFABL code for a module that represents the goal of constantly finding food.}
\end{center}
\label{fig:find-food-code}
\end{figure}

Here we see the value of splitting the world dynamics from the agent module's reward function. We can think of the world and the agent independently. In essence, the definition of a full MDP is split across the definition of a world, and the definition of an agent that acts in that world.

Here we also begin to see syntactic conveniences afforded by the AFABL DSL. There are only two type annotations and one control structure (in the reward function). The rest of the types are inferred by Scala's type inferencer thanks to the way we've written the factory method that creates {\tt AfablModule}s. It's worth noting that we could have refined the DSL to further strip the few Scala syntactic artifacts (like the {\tt if} statement and the anonymous functions for {\tt stateAbstraction} and {\tt moduleReward}) but the syntactic overhead is minimal and there is a tradeoff between writing specific DSL syntax and using Scala's built-in syntax directly. Creating unique syntax for a DSL imposes cognitive burden on programmers who are proficient in the host language. The benefit of the unique syntax must outweigh this cognitive burden. Here we hope to strike the right balance between convenient domain-specific syntax and familiarity to programmers. Thanks to Scala's already concise and experssive language and idioms -- although this looks like a DSL -- there is no special syntax in this example. This code is a good example of shallow DSL embedding.

\subsection{Agents}

An AFABL agent is an agent that acts in a particular world, is composed of independent behavior modules pursuing their own continuing goals, and has a central command arbitrator that uses an agent level reward function to learn when it should listen to each module. As the code in Figure \ref{fig:afabl-bunny-code} shows, an AFABL agent allows programmers to express these components consicely, with very little cognitive distance between the concepts that make up the agent and the code that represents them. As with modules, {\tt AfablAgent} uses features of the Scala programming language to make the syntax more convenient. For example, there is only one explicit type annotation in the AFABL bunny agent code in Figure \ref{fig:afabl-bunny-code}, but behind the scenes a carefully written factory method in the companion object allows Scala's static type inferencer to infer type parameters of the {\tt AfablAgent} constructor, return types for anonymous functions, and assign a concrete type value to a path-dependent abstract type variable. Figuring out all this stuff and wrestling with Scala's type checker directly isn't easy. Writing an AFABL agent is easy.

\begin{figure}[h]
\begin{center}

\begin{lstlisting}[language=Scala]
val bunny = AfablAgent(
  world = new BunnyWorld,

  modules = Seq(findFood, avoidWolf),

  agentLevelReward = (state: BunnyState) => {
    if (state.bunny == state.wolf) 0.0
    else if (state.bunny == state.food) 1.0
    else 0.5
  }
)
\end{lstlisting}

\caption{An AFABL agent that acts in a world, contains behavior modules, and has an agent level reward.}
\end{center}
\label{fig:afabl-bunny-code}
\end{figure}



\subsection{A Complete AFABL Bunny}

A complete bunny agent using the AFABL DSL is shown in Figure \ref{fig:afabl-bunny-code}. This code would typically fit in a single editor window and represents a tremendous amount of functionality. This agent pursues two goals simultaneously and prioritizes them based on the relative locations of the bunny, the food, and the wolf.

\begin{figure}[h]
\begin{center}

\begin{lstlisting}[language=Scala]
val bunnyWorld = new BunnyWorld

case class FindFoodState(bunny: Location, food: Location)
val findFood = AfablModule(
  world = bunnyWorld,
  stateAbstraction = (worldState: BunnyState) => {
    FindFoodState(worldState.bunny, worldState.food)
  },
  moduleReward = (moduleState: FindFoodState) => {
    if (moduleState.bunny == moduleState.food) 1.0
    else -0.1
  }
)

case class AvoidWolfState(bunny: Location, wolf: Location)
val avoidWolf = AfablModule(
  world = bunnyWorld,
  stateAbstraction = (worldState: BunnyState) => {
    AvoidWolfState(worldState.bunny, worldState.wolf)
  },
  moduleReward = (moduleState: AvoidWolfState) => {
    if (moduleState.bunny == moduleState.wolf) -0.1
    else 0.1
  }
)

val bunny = AfablAgent(

  world = new BunnyWorld,

  modules = Seq(findFood, avoidWolf),

  agentLevelReward = (state: BunnyState) => {
    if (state.bunny == state.wolf) 0.0
    else if (state.bunny == state.food) 1.0
    else 0.5
  }
)
\end{lstlisting}

\caption{A complete bunny agent in the AFABL DSL. Code for the modules is repeated from previous figures to give a sense of the full quantity of code requred to write an agent with two behavior modules.}
\end{center}
\label{fig:afabl-bunny-code}
\end{figure}


\section{Validation}

AFABL supports agent-based software abstractions that permit code to be reused in new domains.  This reuse is what we mean by adaptivity: existing AFABL code can adapt to new domains without modifying the code.  We will quantify the value of this reuse in the experiments described below.

\subsection{Experiments}

Programmers were randomly assigned to two equally-sized groups: one group used Scala without AFABL first -- the Scala-first group -- and the other group used AFABL first -- the AFABL-first group.  Each group completed two programming tasks using Scala and AFABL in the order determined by their group.  For each task the programmers were asked to design and implement elegant code that meets the requirements of the task as quickly as possible, balancing the quality of their solutions with time.  The idea was to get a good solution quickly, not a perfect solution in a long time.

\subsubsection{Task 1: The Bunny-Wolf Domain}\label{sec:task1}

\begin{figure}[h]

\begin{center}
\includegraphics[height=2.4in]{bunny.png}
\end{center}


\caption{In the grid world above, the bunny must pursue two goals
  simultaneously: find food and avoid the wolf.  The bunny may move
  north, south, east, or west.  When it finds food it consumes the
  food and new food appears elsewhere in the grid world, when it meets
  the wolf it is eaten and ``dies.''}
\label{fig:bunny-picture}
\end{figure}

In this task each programmers wrote an agent that controls a bunny character in a simple world, depicted in Figure~\ref{fig:bunny-picture}.  The bunny world works as follows:

\begin{itemize}

\item The bunny world is a discrete grid of cells.  The bunny, wolf, and food each occupy one cell.

\item During each time step the bunny may move north, south, east, or west -- this is the bunny agent's action set.

\item Every two time steps the wolf moves towards the bunny.

\item If the bunny moves to the cell currently occupied by the food, the agent should be written to recognize this fact and give the agent an approporate reward signal. In any case the simulation assumes food is ``eaten'' and new food appears elsewhere.

\item If the wolf moves to the cell currently occupied by the bunny it eats the bunny and the bunny ``respawns'' in a new location.

\end{itemize}

Each programmer's Scala bunny and AFABL bunny were run for 1000 time steps and their average scores recorded. The score is based on how much food the bunny eats and how many times the bunny is eaten by the wolf.  Programmers were asked to write bunny agents that meet the wolf as little as possible and eat as much food as possible.

\subsubsection{Task 2: Mating Bunny}\label{sec:task2}

In this task each programmer wrote a bunny agent for a world that is identical to the world in Task 1 except that the bunny must also find mates.  This world includes one static  potential mate that behaves similarly to the food.  When the bunny finds the potential mate, the simulation assumes that the bunny has ``mated,'' the mate disappears, and another potential mate appears elsewhere.  The simulation runs as in Task 1, and the scorer additionally keeps track of how many mates the bunny finds.  As in Task 1, programmers were asked to write bunny agents that meet the wolf as little as possible, eat as much food as possible, and find as many mates as possible.

%% \subsection{Task 3: Adding Wind, Spoiling Food, and Picky Mates}\label{sec:task3}

%% In this task each programmer will write a bunny agent for a world with the same elements as in Task 2 and with the same goals for the bunny, but the world is more complex.  In particular:

%% \begin{itemize}

%% \item There is constant wind from an unchanging direction that affects the wolf's ability to find the bunny.  The wolf will only move toward the bunny if the wolf is downwind of the bunny.

%% \item If food is not eaten within 15 time steps after it appears, it spoils.  Spoilage is represented by the food disappearing and new food appearing elsewhere.

%% \item To simulate selection of fit bunnies, potential mates will only accept the bunny if the bunny has eaten within 10 time steps (a hungry bunny is an unsuccessful bunny and therefore not fit for mating).  Rejection will be represented by the potential mate remaining in place and the bunny not receiving a signal that mating has occurred.

%% \end{itemize}

\subsubsection{Provided Code}

Study participants were given starter code so they couild focus on writing the behavior of their agents. We provided the worlds for each task and the files in which to write their code. Participants were also referred to AFABL documentation on \href{http://afabl.org/}{http://afabl.org/}.

Figure \ref{fig:scala-bunny1} shows the code given to participants for the Scala bunny on Task 1. Figure \ref{fig:afabl-bunny1} shows the code given to participants for the AFABL bunny on Task 1. The {\tt BunnyWorld1}, {\tt BunnyState1}, and {\tt BunnyAction} classes were also provided. It was up to participants to write state abstraction classes if they chose to do so.


\begin{figure}[h]
\begin{center}

\begin{lstlisting}[language=Scala]
class ScalaBunny1 extends Agent[BunnyState, BunnyAction.Value]
    with Task1Scorer {

  // Your code goes in the body of this method. This method defines
  // your agent's behavior, that is, what action it takes in a given
  // state. The last expression in this method must be a
  // BunnyAction.  You may create as many helper functions as you
  // like, but please do not alter any of the provided code.
  def getAction(state: BunnyState, shouldExplore: Boolean = false) = {

    // This is a placeholder to make the code compile. Please
    // replace this with your code.
    BunnyAction.Up
  }
}
\end{lstlisting}

\caption{}
\end{center}
\label{fig:scala-task1-provided}
\end{figure}


\begin{figure}[h]
\begin{center}

\begin{lstlisting}[language=Scala]
object AfablTask1 {

  // Use this val in your agent definitions.
  val bunnyWorld = new BunnyWorld

  // Please place all of your AFABL code for Task 1 in this singleton
  // object.


  // Your solution must assign your AFABL bunny agent for Task 1 to
  // the val afablBuny1.
  val afablBunny1 = ???
}
\end{lstlisting}

\caption{}
\end{center}
\label{fig:afabl-task1-provided}
\end{figure}


The provided code for Task 2 was identical to the provided code for Task 1, except that {\tt BunnyWorld1} was changed to {\tt BunnyWorld2} and {\tt BunnyState1} was changed to {\tt BunnyState2}. For Task 2 participants were encouraged to copy code from Task 1 if they found it helpful or refer to any objects defined for Task 1 that would be helpful, such as behavior modules. As we discuss below, doing so was straightforward for the AFABL agents but not for the Scala agents. Figure \ref{fig:scala-bunny2} shows the code given to participants for the Scala bunny on Task 2. Figure \ref{fig:afabl-bunny2} shows the code given to participants for the AFABL bunny on Task 2. The {\tt BunnyWorld2}, {\tt BunnyState2}, and {\tt BunnyAction} (same as for Task 1) classes were also provided. As in Task 1, it was up to participants to write state abstraction classes if they chose to do so, but they could reuse any state abstraction classes written for Task 1.

Each task had a main method which ran the agents in the worlds to evaluate their performance.

\section{A Typical Task 1 Submission}

\subsection{Scala}

\begin{figure}[h]
\begin{center}

\begin{lstlisting}[language=Scala]

\end{lstlisting}

\caption{}
\end{center}
\label{fig:scala-task1-submission}
\end{figure}


\subsection{AFABL}

\begin{figure}[h]
\begin{center}

\begin{lstlisting}[language=Scala]
  case class FindFoodState(bunny: Location, food: Location)
  val findFood = AfablModule(
    world = bunnyWorld,
    stateAbstraction = (worldState: BunnyState) => {
      FindFoodState(worldState.bunny, worldState.food)
    },
    moduleReward = (moduleState: FindFoodState) => {
      if (moduleState.bunny == moduleState.food) 1.0
      else -0.1
    }
  )

  case class AvoidWolfState(bunny: Location, wolf: Location)
  val avoidWolf = AfablModule(
    world = bunnyWorld,
    stateAbstraction = (worldState: BunnyState) => {
      AvoidWolfState(worldState.bunny, worldState.wolf)
    },
    moduleReward = (moduleState: AvoidWolfState) => {
      if (moduleState.bunny == moduleState.wolf) -0.1
      else 0.1
    }
  )

  val afablBunny1 = AfablAgent(

    world = bunnyWorld,

    modules = Seq(findFood, avoidWolf),

    agentLevelReward = (state: BunnyState) => {
      if (state.bunny == state.wolf) 0.0
      else if (state.bunny == state.food) 1.0
      else 0.5
    }
  )
\end{lstlisting}

\caption{}
\end{center}
\label{fig:afabl-task1-submission}
\end{figure}


As you can see from Figure \ref{fig:afabl-submission}, most AFABL bunny agents look the same. There is one obvious way to implement a bunny agent that must pursue multiple goals. This uniformity is desirable. As Tim Peters says in the Zen of Python, ``There should be one-- and preferably only one --obvious way to do it.'' The similarity in most AFABL solutions to a particular modular agent programming problem is an indication that AFABL provides the right abstractions for adaptive agent programming.

\section{A Typical Task 2 Submission}

\subsection{Scala}

\begin{figure}[h]
\begin{center}

\begin{lstlisting}[language=Scala]

\end{lstlisting}

\caption{}
\end{center}
\label{fig:scala-task2-submission}
\end{figure}


\subsection{AFABL}

\begin{figure}[h]
\begin{center}

\begin{lstlisting}[language=Scala]
  case class FindMateState(bunny: Location, mate: Location)
  val findMate = AfablModule(
    world = bunnyWorld,
    stateAbstraction = (state: BunnyState) => {
      FindMateState(state.bunny, state.mate)
    },
    moduleReward = (state: FindMateState) => {
      if (state.bunny == state.mate) 1.0
      else -0.1
    }
  )

  // Your solution must assign your AFABL bunny agent for Task 2 to
  // the val afablBuny2.
  val afablBunny2 = AfablAgent(

    world = bunnyWorld,

    modules = Seq(AfablTask1.findFood, AfablTask1.avoidWolf, findMate),

    agentLevelReward = (state: BunnyState) => {
      if (state.bunny == state.wolf) 0.0
      else if (state.bunny == state.food) 1.0
      else if (state.bunny == state.mate) 1.0
      else 0.5
    }
  )
\end{lstlisting}

\caption{}
\end{center}
\label{fig:afabl-task2-submission}
\end{figure}

Figure \ref{fig:afabl-task2-submission} shows typical AFABL code for Task 2. Notice that the {\tt findFood} and {\tt avoidWolf} modules from Task 1 have been reused directly. This workds becuase the world, {\tt BunnyWorld}, is the same. In Task 1 the bunny was ignoring the mate. In Task 2 we adapt the bunny to find the mate, and all we need to do is add a {\tt findMate} module and add a line to the {\tt agentLevelReward} function.

\section{Quantitative Results}

We analyzed the submissions of study participants to compare Scala agents to AFABL agents in terms of code size, time spent writing Scala versus AFABL agents, the complexity of Scala verus AFABL agent code, and the proformance of the agents on the assigned tasks.

\subsection{Code Size}

The size of a code base is often correlated with the level of effort required to write or understand the code. We computed the number of lines of code written by study participants for each agent, not including comments or the code that we provided.

We found that ...

\subsection{Time}

Study participants used the IntelliJ IDEA IDE with a plugin that we wrote especially for this study. The plugin recorded timestamps each time the editor tab with the Scala bunny or AFABL bunny gained or lost the focus. We then processed these logs to add up the time deltas between gaining and losing focus as an indication of the time programmers spent writing the code for each bunny agent.

We found that ...

\subsection{Cyclomatic Complexity}

We computed a complexity measure for all the submitted bunny agents. For Scala code we employed the standard McCabe cyclomatic complexity measure.

We found that ...

\subsection{Summary}

Quantitative results for Task 1 are summarized in Table \ref{tbl:task1-results}. Quantitative results for Task 2 are summarized in Table \ref{tbl:task2-results}.

\begin{center}
\begin{table}[h]
\begin{tabular}{|l|r|r|r|}\hline
                      & Scala Mean & AFABL Mean & p-value \\\hline
Lines of Code         & 0.0        & 0.0        & 0.0 \\
Time                  & 0.0        & 0.0        & 0.0 \\
Cyclomatic complexity & 0.0        & 0.0        & 0.0 \\
Cyclomatic complexity & 0.0        & 0.0        & 0.0 \\\hline
\end{tabular}
\caption{Quantitative results of Scala agent code versus AFABL agent code on Task 1. A p-value of less than .05 mean that the difference in means is statistically signifigant at the 95\% significance level, i.e. $H_0: \mu_1 = \mu_2$.}
\label{tbl:task1-results}
\end{table}
\end{center}

\begin{center}
\begin{table}[h]
\begin{tabular}{|l|r|r|r|}\hline
                      & Scala Mean & AFABL Mean & p-value \\\hline
Lines of Code         & 0.0        & 0.0        & 0.0 \\
Time                  & 0.0        & 0.0        & 0.0 \\
Cyclomatic complexity & 0.0        & 0.0        & 0.0 \\
Cyclomatic complexity & 0.0        & 0.0        & 0.0 \\\hline
\end{tabular}
\caption{Quantitative results of Scala agent code versus AFABL agent code on Task 2. A p-value of less than .05 mean that the difference in means is statistically signifigant at the 95\% significance level, i.e. $H_0: \mu_1 = \mu_2$.}
\label{tbl:task2-results}
\end{table}
\end{center}

\section{Qualitative Results}

Programmers responded to a questionnaire to give their impressions of agent programming in AFABL versus agent programming in Scala.

\begin{enumerate}
\item I have a positive impression of agent programming in Scala.

Rationale: programmers’ impression of Scala will provide a baseline for evaluating
programmers’ impression of AFABL.

\item I found it easier to write the agents using AFABL’s programming constructs compared to bare Scala.

Rationale: the point of AFABL is to facilitate agent programming, so programmers should have a more positive impression of AFABL for agent programming.

\item I believe that AFABL facilitated more reusable and maintainable code for agents compared to bare Scala.

Rationale: answers to this question should correlate with answers to Question 1.

\item If given the choice, I would choose AFABL over Scala for agent programming projects.

Rationale: answers to this question should correlate with answers to Question 2.

\item I found it easier to use AFABL compared to Scala for Task 1.

  Rationale: in addition to objective analyses of task submissions, we want to know whether programmers subjectively prefer AFABL.

\item What was it about AFABL that made the Task 1 easier or harder?

Rationale: we want to get open-ended feedback for things we didn’t anticipate.

\item I found it easier to use AFABL compared to Scala for Task 2.

Rationale: in addition to objective analyses of task submissions, we want to know whether programmers subjectively prefer AFABL.

\item What was it about AFABL that made the Task 2 easier or harder?

\end{enumerate}

\section{Adaptive Agent Software Engineering with AFABL}

In this section we discuss the broader implications of adaptive agent programming with AFABL. We define an agent programming space and discuss how AFABL fits into that space.

\subsection{Authorability and the Agent Programming Space}

Authorability is the ease with which a programmer can author the behavior of agents.  Concretely, we suggest a framework for assessing authorability that considers domain knowledge requirements, algorithm knowledge requirements, and the adaptability of agents.

{\bf Domain knowledge} refers to the world-specific details the agent author must program into the agent for a particular domain.  Examples of domain knowledge include representations of state and the dynamics of the world, that is, how actions cause transitions from one state to another.

{\bf Algorithm knowledge} refers to the degree of algorithm detail that must be programmed in the agent.  An agent using a general-purpose programming language with no libraries to support agent programming would need to write the agent's behavior algorithms from scratch. Even an agent using an agent programming library would still likely need to encode a significant amount of algorithm knowledge in the agent. For example, an agent agent that uses a STRIPS planner for action selection would need to contain details of STRIPS operators and the mechanisms for selecting them in response to state perception.

{\bf Adaptability} refers to the ease with which an agent, once authored, can adapt to a changing world or be reused in a different world.

These factors are not completely orthogonal.  High domain knowledge requirements can hinder adaptability because agent agents need to be preprogrammed for worlds that have different dynamics.  Domain knowledge and algorithm knowledge often go hand-in-hand, for example in the encoding of heuristic functions.  Returning to our STRIPS example, STRIPS operators essentially encode world dynamics into the decision making algorithm, thereby coupling domain knowledge and algorithm knowledge.

We say that authorability is high when required domain knowledge is low, algorithm knowledge is low, and adaptability is high.  Such an agent is easy to program in the first place and can be reused in new worlds with minimal reprogramming.

With this working framework for assessing the authorability of agent programming approaches we can map the agent programming space as a spectrum from fully scripted to fully learning approaches.

\subsubsection{Fully scripted agent Programming}

Fully scripted agents are the most common type of agents.  The scripts that control such agents specify every detail of the agent's behavior ahead of time.  While scripted agents can pursue goals and exhibit intelligence, the manner in which these goals are pursued must be written explicitly by the agent author, and if these goals are to be pursued using a particular algorithm, such as a planning algorithm, the algorithm itself must be encoded (or used as a library) from within the code.

In terms of our authorability framework, fully scripted agents have the following properties.

\begin{itemize}
\item Domain knowledge: high. A fully scripted agent must specify a knowledge representation for perception and action that facilitates all the kinds of analyses and decisions the agent will make.  The dynamics of the world must be known in advance and encoded in he script to allow the agent to pursue goals.
\item Algorithm knowledge: high.  Although the algorithms may be simple, such as big if-else ladders, the agent author must have complete knowledge of how the agent's behavior algorithms work.  More complex algorithms mean more complex knowledge for the agent author to manage, fully scripted approaches scale poorly to more complex agents.
\item Adaptability: low.  Once an agent is fully scripted for a given environment, it must be reprogrammed for new environments with different dynamics.  Also, any run-time adaptivity must be scripted explicitly.
\end{itemize}

\subsubsection{Fully Machine Learning agent Programming}

\begin{itemize}
\item Domain knowledge: low. With a sufficiently abstract state representation, the agent can have very little domain knowledge.
\item Algorithm knowledge: low to moderate.  The choice of machine learning algorithm and state representation determine the level of algorithm knowledge necessary to author a fully machine learning agent.
\item Adaptability: high.  Adaptability is the key advantage of machine learning.
\end{itemize}

Neither of these two endpoints of the agent authorability spectrum is desirable.  Fully scripted agents are laborious to write.  Fully machine learning agents typically exhibit a long period of decreasing incompetence until their learning algorithms have sufficient data.  What we want is something in between fully scripted and fully machine learning agents.


{\bf AFABL Hits the Agent Authorability Sweet Spot}

AFABL's integrated reinforcement learning separates the dynamics of the world from the action-slection logic in the agent, freeing the programmer from writing domain-dependent code and facilitating the adaptation of agents to new worlds.

\begin{itemize}
\item Domain knowledge: as much or as little as you want.  You can program the parts you know how to program, and leave AFABL to learn the rest automatically.
\item Algorithm knowledge: moderate.  AFABL is based on the agent and reinforcement learning models.  Behaviors are programmed as actions that execute in response to observed state (hence ``behavior''), and automatic behaviors additionally specify reward signals that enable AFABL to learn the best responses to particular states.
\item Adaptability: high.
\end{itemize}


\chapter{An Example Application: Deriving Behavior from Personality}\label{ch:applications}

In this chapter we present an application of language-integrated reinforcement learning to the problem of personality simulation. Creating artificial intelligent agents that are high-fidelity simulations of natural agents will require that behavioral scientists be able to write code themselves, not merely act as consultants with the ensuing knowledge acquisition bottleneck. However, translating personality models into the concrete behavior of an agent using currently available programming constructs would require a level of code complexity that would make the system inaccessible to behavioral scientists.  What we need is a way to derive the concrete actions of an agent directly from psychological personality models.  This chapter describes a reinforcement learning approach to solving this problem in which we represent trait-theoretic personality models as reinforcement learning agents.  We validate our approach by creating a virtual reconstruction of a psychology experiment using human subjects and showing that our virtual agents exhibit similar behavior patterns. Note that this work was conducted and published before we finished the Arbi-Q command arbitration algorithm of Chapter \ref{ch:arbiq}, so the AFABL agents used the Greatest-Mass q-decomposition algorithm. We have also updated the code from our original work to use AFABL syntax.


\subsection{Introduction}

There is tremendous interest in creating synthetic agents that behave as closely as possible to natural (human) agents.  Rich, interactive intelligent agents will advance the state of the art in training simulations, interactive games and narratives, and social science simulations.  However, the programming systems for creating such rich synthetic agents are too complex, or rather too steeped in computational concepts, to be used directly by the behavioral scientists who are most knowledgeable in modeling natural agents. Engaging behavioral scientists more directly in the authoring of synthetic agents would go a long way towards improving the fidelity of synthetic agents.

What we need is a programming language that a behavioral scientist can use to write agent programs using concepts familiar to behavioral scientists.  This task is complicated by the fact that the most popular and best understood personality models from behavioral science do not lend themselves to direct translation into computer programs.  Requiring a behavioral scientist to specify behaviors in the detail required in even the most cutting edge purpose-built programming language would plunge the would-be behavioral scientist agent programmer right into a morass of complex computational concepts that lie outside the expertise of most dedicated behavioral experts. To solve this problem we need a way to get from personality models to behaviors, to derive specific agent actions in an environment from a personality model without having to program the derivation in great detail.

In this chapter, we describe a way to model motivational factors from trait-oriented personality theory with reinforcement learning modules.  We describe a virtual agent simulation that reconstructs a human subject experiment from psychology, namely some of Atkinson's original work in achievement motivation and test anxiety, and show that our simulation exhibits the same general behavior patterns as the human subjects in Atkinson's experiments.  First, we briefly discuss relevant personality research and provide some background.

\subsubsection{Personality}

Personality is a branch of psychology that studies and characterizes the underlying commonalities and differences in human behavior. Within psychology, there are two broad categories of personality theories: processing theories, and dispositional, or trait theories. Social-cognitive and information-processing theories identify processes of encodings, expectancies, and goals in an attempt to characterize the mechanisms by which people process their perceptions, store conceptualizations, and how those processes drive their interactions with others \cite{dweck1988a-social-cognitive,cervone2009personality,cervone1999the-coherence}. A strength of processing theories, especially from a computational perspective, is that they provide a detailed account of the cognitive processes that give rise to personality and drive behavior.  This strength is also a drawback -- processing theories tend to be detailed and often low-level (though not as low-level as cognitive architectures, which we will discuss below), and this makes them less intuitive and less suited to describing personality in broad, easily understood terms.

Trait theories \cite{cervone2009personality}, the most well-known example of which is the Five-Factor model \cite{mccrae2008handbook}, attempt to identify stable traits (sometimes called ``trait adjectives'') that can be measured on numerical scales and remain invariant across situations in determining behavior.  A strength of the trait approach is that they are well-suited to describing individuals in broad, intuitive terms.  Two drawbacks of the approach are that there is not yet widespread agreement on a set of truly universal traits (or how many there are), and it is not clear how trait models drive behavior.  A promising line of research by Elliot and Thrash \cite{elliot2002approach-avoidance} is working towards solving these problems by integrating motivation into personality in a general way.  The work of Elliot and Thrash particularly supports the approach we present here, as they show that approach and avoidance motivation underpins all currently popular trait theories.

While debate continues about the merits and drawbacks of the different approaches to personality, the psychology community is also attempting to unify personality and motivation theory \cite{mischel2008handbook}. While the work we present here is focused on bridging the gap between the descriptive power of trait-oriented models and the behavior that arise from them, we consider this work to be complementary to work in encoding information processing theories.  In the future, rich computational agents may be built by combining approaches.

\subsubsection{Modeling Personality with Reinforcement Learning}

The essential idea behind modeling personality traits with reinforcement learning is that each motivational factor can be represented by a reinforcement learning module.  In psychology, the inherent desirability or attractiveness of a behavior or situation is referred to as {\em valence}.  For a person high in success approach motivation, behaviors or situations that provide an ``opportunity to excel'' will have high valence, while other behaviors will have lower valence.  The notion of valence translates fairly directly into the concept of reward in reinforcement learning.  Just as people with certain motivational factors will be attracted to high-valence behaviors, a reinforcement learner is attracted to high-reward behaviors.  This is the basis for modeling motivational factors with reinforcement learning modules.  By encoding the valence of certain behaviors as a reward structure, reinforcement learners can learn the behavioral patterns that are associated with particular motivational factors.  This is a powerful idea, because it allows an agent author to write agent code using motivational factors while minimizing the need to encode the complex mechanisms by which such factors lead to concrete behavior.

A critical aspect of trait theory is that traits can have interactive effects.  It is clear that a person who is high in achievement motivation will ``go for it'' when given the opportunity and that a person who is high in avoidance motivation will be more reserved.  But what happens when a person is high in both motivations?  Such interactive effects cannot be ignored in a credible treatment of personality, but it is hard to predict the behavioral patterns that will arise from given combinations of motivational factors.  One can imagine the code complexity that might result from trying to model such interactive effects with production rules or other traditional programming constructs.  As we demonstrate later, our reinforcement learning approach handles such interactive effects automatically.

It is important to note that we are not creating a new theory of personality.  We are creating a computational means of translating existing theories of personality from {\em psychology} (not computer science) into actions executed by synthetic agents.  We are also not committing to a particular theory from psychology, but rather supporting the general category of trait theories of personality which, until now, have not been directly realizable in computer agents.

In the remainder of this chapter we discuss some related work in agent modeling, present our virtual reconstruction of a human subject experiment using our reinforcement learning approach, and discuss the promising results and their implications for future work.


\subsection{Related Work}\label{sec:related-work}

There is a great deal of work in modeling all sorts of phenomena in synthetic agents.  Cognitive architectures provide computational models of many low-level cognitive processes, such as memory, perception, and conceptualization \cite{jones2005an-introduction,langley2008cognitive}.  Cognitive architectures support scientific research in cognitive psychology by providing runnable models of cognitive processes, support research in human-computer interaction with detailed user models \cite{john1998cognitive}, and can serve as the ``brains'' of agents in a variety of contexts.  The most notable and actively developed cognitive architectures are Soar \cite{laird2008extending} and ACT-R \cite{anderson2004an-integrated}.  Recently, some effort has gone into integrating reinforcement learning into Soar \cite{nason2008soar-rl}.  While RL is used to improve the reasoning system in Soar, we are using RL to support new paradigms of computer programming for agent systems.  In general, our work differs from and complements work in cognitive architectures in that we are drawing on psychological theory that is expressed at a much higher level of abstraction.  Cognitive psychology and AI have often built on each other.  Indeed, cognitive psychology is the basis of cognitive architectures in AI.  Our work is an attempt to bring in mainstream personality psychology as a basis for building intelligent agents, which we hope will complement the detailed models of cognitive architectures in creating rich synthetic agents.

There is a large and rich body of work in believable agents.  Mateas and Stern built on the work of the Oz project \cite{loyall1991hap} in creating a programming language and reactive--planning architecture for rich believable agents. They implemented their theory in the computer game Facade, a one-act interactive drama in which the player interacts with computer simulated characters that provide rich social interactivity \cite{mateas2004life-like}.  Gratch, Marsella and colleagues have a large body of work in creating rich simulations of humans for training simulations that incorporate models of appraisal theory and emotion \cite{gratch2005lessons,swartout2006toward}.  A distinctive feature of the work of both Mateas, et. al., and Gratch, et. al., is that they are dealing with the entire range of AI problems in creating believable agents that sense, act, understand and communicate in natural language, think, and exhibit human-like personalities.  Our work differs from other work in personality modeling in that we are not attempting to simulate personality, but using definitions of personality to drive the behavior of synthetic agents.  We want to derive behavior that is consistent with a given personality model, but not necessarily to ensure that the agent gives the appearance of having that personality.


\subsection{Experiments}

To test our claim that personality can be modeled by reinforcement learning modules, we created a population of simple two-module multiple-goal reinforcement learning agents and ran them in a world that replicated experiments carried out with humans by psychologist John Atkinson.  First we describe Atkinson's original research, and then discuss our virtual reconstruction of his experiments.

\subsubsection{Atkinson's Ring Toss Experiment}\label{sec:ring-toss}

John Atkinson was among the first researchers to study the existence and role of approach and avoidance motivation in human behavior. Prior to Atkinson's work, it was believed that test anxiety was equivalent to low achievement motivation.  However, Atkinson showed that test anxiety is actually a separate avoidance motivation, a ``fear of failure'' dimension that works against and interacts with achievement motivation \cite{atkinson1960achievement}.  To test his hypothesis, he administered standard tests of achievement motivation and test anxiety to a group of undergraduate psychology students and devised a series of experiments which examined the effort put forth in achieving success in tasks such as taking a final exam.  It is important to note that he did not measure the outcomes of the task, but rather the effort put forth in doing well in them.  Thus, his experiments examined the relationship between motivation and {\em behavior}, not necessarily competence.  One of his experiments, a ring toss game, produced results that clearly show the interplay of approach and avoidance motivation and is particularly well-suited to computer simulation.

In Atkinson's ring toss experiment, subjects played a ring toss game in which players attempted to toss a ring from a specified distance onto a peg.  Subjects made 10 tosses from any distance they wished, from 1 through 15 feet, and the distance at which each subject made each toss was recorded.  For analysis, subjects were divided into four groups according to their measures of achievement motivation and test anxiety so that the relationship between these motivations and their behavior could be analyzed.  For each of the two measures -- achievement motivation and test anxiety -- subjects were classified as either high or low, with the dividing line between high and low set at the median scores in each measure.  (For example, a H-L subject is high in achievement motivation and low in test anxiety).  Subjects were divided into four groups -- H-L, H-H, L-L, and L-H -- and the percentage of shots taken at each distance by each group was recorded. We discuss his results and our simulation below.

\subsubsection{Computational Models of Atkinson's Subjects}

We reconstructed Atkinson's ring toss experiment in a computer simulation.  We created 49 virtual agents that corresponded to each of the 49 human subjects in Atkinson's experiments, with the same distribution of high and low measures of achievement motivation and test anxiety.  Simplified code for a representative student subject is presented in Figure \ref{fig:student}.  Since we did not have access to Atkinson's source data, we modeled high motivation measures as having a mean of 1.5 and low motivation with a mean of 0.5, both with standard Normal distributions (mean = 0, variance = 1) scaled by $\frac{1}{2}$, so virtual test subjects did not all have the same measures.

\begin{figure}[h]

\begin{lstlisting}
val motivatedStudent = GmAgent(
  world = RingTossWorld,

  // Sequence of pairs where the second element of each pair
  // is the weight of the pair, corresponding to the personality
  // trait measure
  modules = Seq((achievementMotivation, 1.5 + X ~ N(0, 1) / 2),
                (testAnxiety, .5 + X ~ N(0, 1) / 2))
}
\end{lstlisting}

\caption{An agent representing a success-oriented student in Atkinson's ring toss experiment, containing two RL modules representing high achievement motivation and low test anxiety.  The code snippets presented here are simplified versions of the Scala code we used to run our experiments.}
\label{fig:student}
\end{figure}

As discussed earlier, each of the motivational dimensions of the virtual subjects was implemented with reinforcement learning modules that learned to satisfy the preference for perceived valence of behaviors (modeled as reward).  For example, in the achievement motivation module (see Figure \ref{fig:achievement}), the greater the distance from the peg, the greater the reward because it represents greater achievement.  Similarly, in the test anxiety module (see Figure \ref{fig:testanxiety}), greater reward is given to closer distances, because they minimize, or ``avoid'' the chance of failure from a long-distance toss.

\begin{figure}[h]
\begin{lstlisting}
val achievementMotivation = AfablModule(

  world = RingTossWorld,

  moduleReward = (state: RingTossState) => state match {
    case OneFootLine => 1,
    case TwoFootLine => 2,
    ...
    case FifteenFootLine => 15
  }

)
\end{lstlisting}
\caption{A reinforcement learning module representing achievement motivation.}
\label{fig:achievement}
\end{figure}

\begin{figure}[h]

\begin{lstlisting}
val testAnxiety = AfablModule(

  world = RingTossWorld,

  moduleReward = (state: RingTossState) => state match {
    case OneFootLine => 15,
    case TwoFootLine => 14,
    ...
    case FifteenFootLine => 1
  }

)
\end{lstlisting}

\caption{A reinforcement learning module representing Test Anxiety (`avoidance motive, a.k.a. ``fear of failure'').  Note that the rewards are inverted from the achievement motivation module, that is, the valence of avoiding achievement is higher.}
\label{fig:testanxiety}
\end{figure}

Internally, each personality module is implemented with the standard Q-learning algorithm \cite{sutton1998reinforcement}.  The ring toss world consists of 16 states -- a start state and one state for each of the 15 distances, and 15 actions available in each state that represent playing (making a toss) from a particular distance.  Each reinforcement learning module used a step-size parameter of $\alpha = 0.1$, a discount factor of $\gamma = 0.9$ (though discounting wasn't important given that the 15 states representing playing lines were terminal states, since each play was a training episode), and employed an $\epsilon$-greedy action selection strategy with $\epsilon = 0.2$.  (Readers familiar with reinforcement learning will also notice that this game is equivalent to a 15-armed bandit problem.)  We emphasize that the details of the reinforcement learning algorithms are not essential to modeling motivational factors, and those details are hidden inside the implementation of the modules.  Indeed a major goal of our work is to simplify the task of writing synthetic agents by taking care of such details automatically.

Recall that reinforcement learning algorithms learn an action value for each action available in a given state.  An action value for a state represents the expected total reward that can be achieved from a state by executing that action and transitioning to a successor state. For each of the modules -- Achievement and TestAnxiety -- the action values represent the learned utility of the actions in serving the motivational tendencies the modules represent.  The Student agents take into account the preferences of the modules -- represented by action values -- by summing their action values weighted by their module weights to get a composite action value for each action in a given state.  If we denote each module's action value by $Q(s, a)$ and the weights by $W$, then the composite, or overall, action value is:

\begin{align}
Q_{student}(s,a) =  & W_{Achievement} Q_{Achievement}(s,a) +\\
                   & W_{TestAnxiety} Q_{TestAnxiety}(s,a)
\end{align}

For the virtual experiments, each module -- Achievement and TestAnxiety -- was run to convergence and then the student agents simulated 10 plays of the ring toss game, just as in Atkinson's experiment.  We discuss the results of the experiment below.

\subsection{Model Validation}

%% 8< 8< 8< 8< 8< 8< 8< 8< 8< 8< 8< 8< 8< 8< 8< 8< 8< 8< 8< 8< 8< 8< 8<

%% When stats are done, gonna scrap this in favor of the table containing
%% confidence intervals.

%% \begin{figure}[h]
%%   \begin{center}
%%     \scalebox{.8}{\includegraphics{atkinson}}
%%     \scalebox{.4}{\includegraphics{iccm}}
%%     \caption{The top plot shows the behavior patterns of human
%%       subjects in Atkinson's Ring Toss experiment.  The bottom plot
%%       shows the behavior patterns of our synthetic agents that
%%       re-created Atkinson's experiment.  Note that Atkinson's plot is
%%       smoothed, while ours is not.}
%%   \end{center}
%%   \label{fig:results}
%% \end{figure}

%% As Figure \ref{fig:results} shows, our synthetic agents exhibit
%% behavior patterns that are very similar to those of the human subjects
%% in Atkinson's original experiments, strongly supporting our claim that
%% motivational factors can be modeled with reinforcement learning
%% modules.

%% 8< 8< 8< 8< 8< 8< 8< 8< 8< 8< 8< 8< 8< 8< 8< 8< 8< 8< 8< 8< 8< 8< 8<

A model is a set of explicit assumptions about how some system of interest works \cite{law2007simulation}.  In psychology the system of interest is (usually) a human or group of humans.  Our virtual reconstruction of Atkinson's experiments constitutes a computational representation of Atkinson's two-factor model of personality.  Thus, our agents are simulation models of Atkinson's subjects (the students in his ring toss experiment).  While the work presented here is only a proof of concept, we do hope to achieve a high level of validity as we refine our approach, so it will be useful to validate our models using techniques from simulation science \cite{law2007simulation}.

As we described earlier, Atkinson divided his subjects into four groups according to their measures (high or low) on achievement motivation and test anxiety.  For each of these four groups -- H-L, H-H, L-L, L-H -- he recorded the percentage of shots that each group took from each of the 15 distances.  We ran 10 replications of our simulation and recorded the mean percentages for each group and distance.  For each percentage mean we calculated a 95\% confidence interval.  We consider a model to be valid if the confidence intervals calculated on the simulation percentage means contain the percentages obtained by Atkinson in his experiments with human subjects.

\begin{table*}[ht]
\begin{center}

\begin{tabular}{|l||c|c|c|c|} \hline
Achievement: & High & High & Low & Low \\
Test Anxiety: & Low & High & Low & High \\  \hline
 & Atkinson & Atkinson & Atkinson & Atkinson \\
 & Simulation & Simulation & Simulation & Simulation \\
Range & Conf. Int. & Conf. Int. & Conf. Int. & Conf. Int. \\ \hline\hline
  1-7 & \bf{11}           & 26              & 18           & 32 \\
      & \bf{7.7}          & 14.0            &  5.6         &  8.5 \\
      & \bf{(4.0, 11.4)}  & (5.6, 22.4)     & ( 1.4,  9.7) & ( 4.4, 12.5)\\ \hline
8-12  & \bf{82}           & 60               & 58           & 48 \\
      & \bf{75.4}         & 69.0             & 74.4         & 80.0 \\
      & \bf{(65.1, 85.7)} & (61.1, 76.9)     & (62.0, 86.9) & (74.1, 85.9)\\ \hline
13-15 & 7                 & \bf{14}           & \bf{24}           & 20 \\
      & 16.9              & \bf{17.0}         & \bf{20.0}         & 11.5 \\
      & ( 8.8, 25.0)      & \bf{(9.4, 24.6)}  & \bf{(8.3, 31.7)} & ( 6.9, 16.2)\\ \hline
\end{tabular}
\caption{Validation Results.  For each subject group the percentage of shots taken by Atkinson's human subjects and by our simulation from each of three ranges is presented along with a 95\% confidence interval for the mean percentage of shots in 10 simulated replications of Atkinson's experiment.}
\label{tab:results}

\end{center}
\end{table*}

The validation results are presented in Table \ref{tab:results}. Atkinson analyzed his experimental data by aggregating the shots taken by subjects into three ``buckets'' representing low, medium, and high difficulty.  In Atkinson's analyses the dividing lines between the three buckets were set in four different ways with each yielding similar results.  For brevity we present the division obtained by using both geographical distance and distribution of shots about the median shot of 9.8 ft, in other words, the dividing line one would choose by inspecting the histogram for distinct regions.  This strategy resulted in the three buckets listed in the left column of Table \ref{tab:results}.  Each cell of the four subject groups -- H-L, H-H, L-L, L-H - contains the percentage of shots taken by Atkinson's subjects, the mean percentage obtained by running 10 replications of our simulation of Atkinson's experiment, and a 95\% confidence interval for the mean percentage.  While our model did not achieve formal validation, the general patterns of behavior are quite similar to Atkinson's human subject experiment and we consider these results to be a good proof of concept.  We discuss some reasons behind these results and strategies for improvement below.

\subsection{Discussion}

We made several assumptions in our models that affected the validation results.  First, because we did not have access to Atkinson's original data, only summary presentations, we did not know the exact distribution of motivational factors among his subjects, or even the scales used in his measures.  We assumed normally distributed measures and tried several different scales before settling on the values used in the simulations reported here.  Second, it is not clear how the valence of behaviors should be translated into reward structures for RL agents.  We chose a simple linear reward structure in hopes that the system would be robust to naive encodings.  To make our approach widely useful we will need to address the manner in which reward structures are determined.

%% Third, we calculated aggregate action values by a simple weighted sum of module action values.  We are currently investigating optimal arbitration of multiple RL modules and hope to report results within the next six months.

We chose the Atkinson ring toss experiment on the advice of psychologists who recommended it as a well-known example of trait-oriented behavior theory, and because of its simplicity. However, our goal is to create large agent systems, so future work will need to address scalability -- to greater numbers of trait factors and more complex worlds -- and generalizability, or transferability, to other domains.

%% More generally, there are broader issues to be understood about
%% employing this method in practice.  From the psychology perspective, we
%% need to understand how motivational factor scales translate into the
%% weights on reinforcement learning modules, e.g., they must be
%% normalized somehow.  This issue is similar to the issue of reward
%% comparability in reinforcement learning (discussed below).  Also, our
%% example used two motivational factors.  How do we scale this approach
%% to multiple factors?  Many trait theories contain fewer traits that
%% subsume traits in other, more detailed trait theories.  We would
%% expect such decompositions to be helpful in our formulation as well.

%% The reinforcement learning algorithm employed in these experiments
%% were standard Q-learners.  While these algorithms worked well on this
%% limited problem, they likely will not be appropriate for every
%% problem.  Indeed the No Free Lunch Theorem \cite{ho2001simple} informs
%% us that each task must be matched to an appropriate algorithm -- there
%% is no universal solution to every problem.  How do we characterize the
%% matching of reinforcement learning algorithms to particular agent
%% designs and virtual worlds?

%% The algorithms we used also employed no optimization.  Reinforcement learning suffers from the curse of dimensionality, and many techniques are being actively pursued to cope with the size of state spaces for realistic-size domains.  Profitably employing reinforcement learning in agent programming systems will mean integrating scaling techniques such as function approximation (e.g., of action-value functions or state spaces) and decomposition techniques.

Finally, notice that the example code presented in this paper contains no logic for implementing behavior.  The agents and the modules are defined declaratively by specifying a state space, an action set, and a reward structure.  The run-time system derives the concrete behavior of the agents automatically from these specifications.  This technique, sometimes called partial programming or adaptive programming\cite{simpkins2008towards}, is a key concept that increases the usability of agent programming by allowing programmers to specify {\em what} an agent is to do without getting mired in {\em how} the agent should do it.

\subsection{Conclusions and Future Work}

Much work remains to make accessible personality-based agent programming systems a reality, and our work is progressing on three paths.  First, the integration of reinforcement learning into agent programming systems needs to be studied further so that we know when it is useful and how much detail can be hidden from the agent programmer. This dissertation has confirmed what we already know, namely, that authoring reward functions is not straightforward. We need to be able to specify modules in simpler terms and let the reward structure be derived automatically (we will discuss this further in Chapter \ref{ch:conclusion}. Second, the examples presented here were written together so that the reward signals of each agent were directly comparable, which allowed us to use the Greatest-Mass q-decomposition algorithm for combining the modules. Now that we have an arbitration algorithm that is robust to incomparable reward scales, we can either use a {\tt GmAgent} for personality modeling, as we have done here, or extend AFABL with weighting to enable trait-oriented personality modeling. Finally, while AFABL is currently able to handle the personality modeling presented in this chapter, AFABL is still a shallowly-embedded Scala DSL and therefore beyond the programming capabilities of most psychologists. We will need to make AFABL simpler to use. Nevertheless, reinforcement learning provides a promising approach to modeling personality traits and motivational factors in synthetic agents.  In particular, it provides us with a means to create agent programming systems that are at least comprehensible by behavioral scientists and harness their knowledge directly while minimizing the need for complex programming.


\chapter{Conclusion}\label{ch:conclusion}

\section{Review of Major Contributions}

This dissertation has reported on two primary contributions: a command arbitration algorithm for robust modular reinforcement learning, and a domain-specific language that integrates modular reinforcement learning, AFABL. The benefits of langauge-integrated reinforcement learning have been demonstrated in a study of programmers using AFABL compared to using a traditional programming language for the same tasks. AFABL agents are easier to write, are expressed in less complex code, and have more readily reused components than agents written in traditional programming languages. In the remainder of this chapter we discuss some implications of AFABL (as a representative first step in language-integrated modular reinforcement learning), limitations of the current work, and directions for future work.

\section{Adaptive Agent Software Engineering with AFABL}

In this section we discuss the broader implications of adaptive agent programming with AFABL. For the present discussion we define authorability is the ease with which a programmer can author the behavior of agents.  Concretely, we suggest a framework for assessing authorability that considers domain knowledge requirements, algorithm knowledge requirements, and the adaptability of agents.

{\bf Domain knowledge} refers to the world-specific details the agent author must program into the agent for a particular domain.  Examples of domain knowledge include representations of state and the dynamics of the world, that is, how actions cause transitions from one state to another.

{\bf Algorithm knowledge} refers to the degree of algorithm detail that must be programmed in the agent.  An agent using a general-purpose programming language with no libraries to support agent programming would need to write the agent's behavior algorithms from scratch. Even an agent using an agent programming library would still likely need to encode a significant amount of algorithm knowledge in the agent. For example, an agent agent that uses a STRIPS planner for action selection would need to contain details of STRIPS operators and the mechanisms for selecting them in response to state perception.

{\bf Adaptability} refers to the ease with which an agent, once authored, can adapt to a changing world or be reused in a different world.

These factors are not completely orthogonal.  High domain knowledge requirements can hinder adaptability because agent agents need to be preprogrammed for worlds that have different dynamics.  Domain knowledge and algorithm knowledge often go hand-in-hand, for example in the encoding of heuristic functions.  Returning to our STRIPS example, STRIPS operators essentially encode world dynamics into the decision making algorithm, thereby coupling domain knowledge and algorithm knowledge.

We say that authorability is high when required domain knowledge is low, algorithm knowledge is low, and adaptability is high.  Such an agent is easy to program in the first place and can be reused in new worlds with minimal reprogramming.

With this working framework for assessing the authorability of agent programming approaches we can map the agent programming space as a spectrum from fully scripted to fully learning approaches.

\subsubsection{Fully scripted agent Programming}

Fully scripted agents are the most common type of agents.  The scripts that control such agents specify every detail of the agent's behavior ahead of time.  While scripted agents can pursue goals and exhibit intelligence, the manner in which these goals are pursued must be written explicitly by the agent author, and if these goals are to be pursued using a particular algorithm, such as a planning algorithm, the algorithm itself must be encoded (or used as a library) from within the code.

In terms of our authorability framework, fully scripted agents have the following properties.

\begin{itemize}
\item Domain knowledge: high. A fully scripted agent must specify a knowledge representation for perception and action that facilitates all the kinds of analyses and decisions the agent will make.  The dynamics of the world must be known in advance and encoded in he script to allow the agent to pursue goals.
\item Algorithm knowledge: high.  Although the algorithms may be simple, such as big if-else ladders, the agent author must have complete knowledge of how the agent's behavior algorithms work.  More complex algorithms mean more complex knowledge for the agent author to manage, fully scripted approaches scale poorly to more complex agents.
\item Adaptability: low.  Once an agent is fully scripted for a given environment, it must be reprogrammed for new environments with different dynamics.  Also, any run-time adaptivity must be scripted explicitly.
\end{itemize}

\subsubsection{Fully Machine Learning agent Programming}

\begin{itemize}
\item Domain knowledge: low. With a sufficiently abstract state representation, the agent can have very little domain knowledge.
\item Algorithm knowledge: low to moderate.  The choice of machine learning algorithm and state representation determine the level of algorithm knowledge necessary to author a fully machine learning agent.
\item Adaptability: high.  Adaptability is the key advantage of machine learning.
\end{itemize}

Neither of these two endpoints of the agent authorability spectrum is desirable.  Fully scripted agents are laborious to write.  Fully machine learning agents typically exhibit a long period of decreasing incompetence until their learning algorithms have sufficient data.  What we want is something in between fully scripted and fully machine learning agents.


{\bf AFABL Hits the Agent Authorability Sweet Spot}

AFABL's integrated reinforcement learning separates the dynamics of the world from the action-selection logic in the agent, freeing the programmer from writing domain-dependent code and facilitating the adaptation of agents to new worlds.

\begin{itemize}
\item Domain knowledge: as much or as little as you want.  You can program the parts you know how to program, and leave AFABL to learn the rest automatically.
\item Algorithm knowledge: moderate.  AFABL is based on the agent and reinforcement learning models.  Behaviors are programmed as actions that execute in response to observed state (hence ``behavior''), and automatic behaviors additionally specify reward signals that enable AFABL to learn the best responses to particular states.
\item Adaptability: high.
\end{itemize}


\section{Limitations of Current Work}

% The work presented here is promising, but focused in scope.

\subsection{Reward Authoring}

As many other researchers have noted, reward authoring is not straightforward for programmers not trained in reinforcement learning. Study participants spent much of their AFABL writing time trying out different reward structures in an effort to improve their agents' performance. Although we provided documentation with hints on how to author reward structures, writing good reward functions is too opaque for most programmers. In the next section we discuss a possible improvement to AFABL which would relieve programmers from writing the reward functions of modules.

\subsection{Training}

Using any reinforcement learning-based programming system requires the availability of a simulation environment to train the learning modules before being used ``in production.'' Using an untrained reinforcement learning agent and accepting that it will perform poorly until it learns is not practical because reinforcement learning algorithms typically require hundreds or thousands of iterations to reach an acceptable level of performance. Separate modules with local state abstractions and reward functions help speed up training, but finding good factorizations into modules is a potentially steep burden to place on the programmer for larger agents.

\subsection{Host Language Limitations}

Writing AFABL agents is writing Scala code, so AFABL programmers must have at least basic competence with Scala and the Scala tool chain. Since it was outside the scope of the present work, we did not try to determine how much of the Scala tooling can be hidden or automated for AFABL programmers. We required study participants to use a recent version of IntelliJ IDEA and provided a pre-configured Scala/AFABL project and an IntelliJ plugin to automate the time tracking and submission process. Still, several participants had trouble running the study code smoothely, as is often the case with development tools. Many study participants who did not participate in a group session simply abandonded the study. We advertised the study to the Atlanta Scala Meetup, a group of local software engineers either using Scala professionally or interested in learning. Approximately 15 Scala Meetup members started the study and only one finished. Due to the amount of individual software issues I needed to help participants solve -- differing operationg systems, IntelliJ versions, etc -- I believe many of these dropouts were due to simple software setup issues.

In addition to Scala tooling issues, AFABL programmers must deal with the Scala programming language. For example, when programmer makes a mistake in their code the error messages come from the Scala compiler and run-time system, not AFABL. Luckily, in the study few people had such issues with AFABL code itself. With more complex agents problems are more likely to occur, and the programmer may be faced with the famous complexity of the Scala type system. The AFABL programmer who is not also a competent Scala programmer has little hope of debugging non-trivial errors.

\section{Directions for Future Work}


\subsection{Refined Module Types}

AFABL currently supports a narrow definition of an agent: a behaving entity with a set of states that must constantly be pursued or avoided. In reinforcement learning these kinds of modules are called called continuing tasks, as opposed to episodic goals. Previous version of AFABL supported a greater set of features but we removed them to focus on AFABL's core for the purpose of this work. With a cleaner core AFABL we could re-implement the features discussed below.

\paragraph{Drives}

A Drive is a behavior module that runs throughout the life of its conataining agent and represents a state that an agent should constantly seek.

\paragraph{Aversions}

An Aversion module is a behavior module that runs throughout the life of its conataining agent and represents a state that an agent should constantly avoid.  It is a constraint in the sense that, in certain states, a constraint module will identify actions that should *not* be executed.

\paragraph{Objectives}

A an objective is a short-term goal state that generates a drive module that is active until its goal is achieved.  The command arbitrator gives objective modules priority over drive modules, but all modules are constrained by constraint modules.

\paragraph{Tasks}

A task is a temporally-extended action, a "mini-policy" that acheives a subgoal.  Tasks are equivalent to subtasks (MaxQ), abstract machines (PHAM), or options from hierarchical reinforcement learning. Tasks could be manually authored, or algorithms from hierarchical reinforcement learning could be integrated into AFABL.

\subsection{Simplified Syntax}

The features listed above may make it possible to automatically author reward functions for modules. For example, the Bunny agent for Task 2 from Chapter \ref{ch:afabl} could be written with Drives and Aversions as follows:

\begin{lstlisting}[language=Scala]
  val afablBunny2 = AfablAgent(

    world = bunnyWorld,

    drives = Drives(state: BunnyState) {
      (BunnyState.bunny == BunnyState.food),
      (BunnyState.bunny == BunnyState.mate)
    },

    aversions = Aversions(state: BunnyState) {
      (BunnyState.bunny == BunnyState.wolf),
    }

    agentLevelReward = (state: BunnyState) => {
      if (state.bunny == state.wolf) 0.0
      else if (state.bunny == state.food) 1.0
      else if (state.bunny == state.mate) 1.0
      else 0.5
    }
  )
\end{lstlisting}

Instead of writing code to specify modules, the programmer specifies states that are to be constantly sought or avoided -- expressed as state predicates -- and the modules are derived from them automatically. Note also that this proposed syntax does not include state abstraction functions in modules becuase they could be derived automatically from the states that are to be sought or avoided.

\subsubsection{Drama Manager Support}

The features discussed above would go along way toward supporting drama managers for intelligent interactive narratives. In addition, a drama manager would need to be able to activate and deactivate modules and inject new objectives to support particular story goals.

\subsection{General Agent Architecture}

The current version of AFABL focuses on integrated reinforcement learning but could easily be extended to support integrated intelligence, that is, mixing of agent modules that employ differnt kinds of AI algorithms. Because an AFABL agent performs command arbitration over modules that support a behavioral interface (providing an action given a state observation) as opposed to merging elements of reinformcenet learners (like Q-values), the modules themselves can employ any mechanism to decide on actions given a state.  This information hiding means that AFABL agents could be composed of a mixture of modules that use many different kinds of AI, including statistical learning, rule-based reasoning, or (reactive) planning.  In this sense AFABL would be an integrated intelligence architecture.

\paragraph{Knowledge-Based Arbitrators}

In addition to the modules the arbitrator itself could employ different kinds of algorithms for command arbitration. A knowledge-based arbitrator could use hand-coded logic to decide from among the actions recommended by an agent's modules.  Simple arbitrators with few modules to arbitrate can often be coded quite simply as knowledge-based arbitrators.

\paragraph{Hierarchical Decomposition}

Beacuse modules are themselves agents, modules can contain other modules and perform command arbitration over those modules just as the top-level agent does.  Agents can thus be decomposed recursively into behavioral subsystems.  This recursive behavior module decomposition would provide the agent designer with great flexibility.  Recursive module composition is somewhat similar to the levels of competence in Brooks's subsumption architecture with an important difference: the internal workings of modules are never altered externally.  Modules are treated as black-boxes.  Command arbitration accomplishes the same result that output suppression does in classic subsumption.

\subsection{Independent (Non-Embedded) Language}\label{sec:conclusion-full-language}

Finally, once the additions to the language are integrated into the internal DSL and studied and refined sufficiently, an external DSL could be considered. Although an external DSL is far more work to implement, the benefits could justify the cost. A stand-alone version of AFABL would have its own set of development tools, report agent-oriented error messages to the user, and potentially run faster than equivalent internal DSL code.


\appendix


\begin{postliminary}
\references
\postfacesection{Index}{%
%%             ... generate an index here
%%         look into gatech-thesis-index.sty
}
\begin{vita}

\section{Education}

\begin{tabular}{lllr}
%{\bf Degree} & {\bf Year} & {\bf University}\hspace{2.0in} & {\bf Field} \\
%\hline \\[\tabitemskip]
\textbf{Ph.D.} & 2016 (expected) & Georgia Institute of Technology &
{\sl Computer Science}\\
\textbf{M.S.} &  2004 & Southern Polytechnic State University &
{\sl Computer Science}\\
\textbf{B.S.} & 1990 & United States Air Force Academy & \\
\end{tabular}

\section{Employment History}

\begin{tabular}{llr}
%\textbf{Title} & \textbf{Organization} & \textbf{Years} \\ \hline \\[.1in]

\textbf{Research Scientist II} & Georgia Institute of Technology
                               &{\sl 2001-present}\\
                            & Atlanta, GA & \\
\textbf{Software Engineer} & Internet Security Systems & {\sl
  2000-2001}\\
                           & Atlanta, GA & \\
\textbf{Software Engineer,} & U.S. Air Force & {\sl 1998-2000}\\
\textbf{IT/Network Manager} & Columbus AFB, MS & \\
\textbf{T-37 Instructor Pilot} & U.S. Air Force & {\sl 1997-2000}\\
                           & Columbus AFB, MS & \\
\textbf{KC-135 Pilot} & U.S. Air Force & {\sl 1995-1997}\\
                           & McConnell AFB, KS & \\
\textbf{Interactive Courseware} & U.S. Air Force & {\sl 1993-1995}\\
\textbf{Developer} & Vandenberg AFB, CA & \\
\textbf{Space Instructor} & U.S. Air Force & {\sl 1992-1995}\\
                           & Vandenberg AFB, CO; Lowry AFB, CO & \\
\textbf{Student Pilot} & U.S. Air Force & {\sl 1990-1992}\\
                           & Columbus AFB, MS &
\end{tabular}

\end{vita}
\end{postliminary}
\end{document}
