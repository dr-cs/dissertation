\chapter{Introduction}

This chapter sets the stage for the work presented in this dissertation.

\section{The Promise of AI and the Challenges of Software Engineering}

Artificial intelligence was one of the first grand promises of computing.

On the other hand, the rest of computer science went about the business of creating hardware and software systems for a variety of applications, initially business management and scientific computing.  Since the personal computing revolution, a great deal of effort has been put into making software systems easy for all konds of users, technical and non-technical.  Through it all, computing power has increased dramatically and the software systems users demand have exploded in size and complexity.  Software engineering has struggled to keep pace with the growing size and complexity of these systems. Today the field of software engineering, both form academia and industry, has developed a well-defined set of practices and design guidelines that result in software systems that are maintainable,

\subsection{Reinforcement Learning}

One can think of reinforcement learning (RL) as a machine learning approach to planning, that is, a way of finding a sequence of actions that achieves a goal.  The RL problem formulation is this: an agent's world is described by a set of states, the agent can execute one of a set of actions in each state, and the agent is rewarded to greater or lesser degrees for each state-changing action it executes. In RL, problems of decision-making by agents interacting with uncertain environments are usually modeled as Markov decision processes (MDPs). In the MDP framework, at each time step the agent senses the state of the environment and executes an action from the set of actions available to it in that state. The agent's action (and perhaps other uncontrolled external events) cause a stochastic change in the state of the environment. The agent receives a (possibly zero) scalar reward each time it executes an action and makes a transition to a new state. The agent's goal is to find a {\it policy} that says which action should be chosen in each state.  The policy should specify actions that maximize the expected sum of rewards over some time horizon. An optimal policy is a mapping from states to actions that maximizes the long-term expected reward.  In short, a policy defines which action an agent should take in a given state to maximize its chances of reaching a goal.  Reinforcement learning is a large and active area of research, but the preceding is all the reader needs to understand the work presented here.  More detail can be found in \cite{sutton1998reinforcement,kaelbling1996reinforcement}.

\subsection{Software Engineering}



\subsection{Programming Langauges}



\section{Contributions}

The primary contribution of this work is to marry AI and software engineering in a way that advances both fields. The needs of practical software engineering inspires a new AI algorithm for modular reinforcement learning. Integrating this new formulation of MRL and associated algorithms in to a programming language enables a new kind of software engineering: modular agent programming.


\section{Overview}
