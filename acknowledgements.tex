\clearpage
\begin{centering}
\textbf{ACKNOWLEDGEMENTS}\\
\vspace{\baselineskip}
\end{centering}

%Insert your dedication text here

First I must thank my parents. My mother was there for me every day. My father worked multiple jobs to support his family and taught me to work hard for everything I received. Without their love and support I would not have taken this path.

I started my Ph.D. journey over a decade ago while working at Georgia Tech Research Institute. Ashwin Ram, my first advisor, guided me through the Ph.D. application and gave me some of my earliest research opportunities. Sheila Isbell connected me to the right lab for me at GTRI, ITTL (now ICL). Terry Hilderbrand and Margaret Loper helped me access GTRI's Ph.D. support, and supported me as I balanced my GTRI duties with doctoral studies.

Charles Isbell accepted me as his student just when I needed him. Becoming Charles's student and joining his lab was invigorating. Just working near, conversing with, and meeting with Michael Holmes, Chip Mappus, David Roberts, Peng Zang, Sooraj Bhat, Luis Carlos ``Luisca'' Cobo Rus, Liam Mac Dermed, Arya Irani, Mark Nelson, Jon Scholz, Ryan Curtin, Kaushik Subramanium, Pushkar Kolhe, Ashley Edwards and Himanshu Sahni did a great deal to shape my development as a researcher.  And they were as warm as they were brilliant. My children have as many fond memories of my lab mates as I do, from Nerf fights to cookouts to chess matches to ``funking out'' Charles's office when he was awarded tenure.

My work builds on the earlier work of my lab mate and friend, Sooraj Bhat. Sooraj helped me get started in language-integrated reinforcement learning and shared my interest in advanced programming languages.

As a graduate student I was inspired by Ashok Goel's Artificial Intelligence and Cognitive Science classes, Charles's Machine Learning class, and Spencer Rugaber's Programming Language Design class.

\newpage

I am grateful that Mark Riedl and Charles invited me to join their funded project on computational interactive narratives. Working with them was a joy. After that project ended, Charles and my current boss, Bill Leahy, gave me the opportunity to become a lecturer. I am grateful to Bill for welcoming me into a group of outstanding teaching professionals that will likely be my home for the rest of my professional life.

I did most of my writing during Fall 2016. My CS 1331 Head TA, Taylor Hartman, took on a great deal of responsibility to allow me more time to write. CS 1331 could not have functioned so well without her leadership and the groundwork laid by past Head TAs like Keith Cartledge, Thomas Shields, Aaron Friesen, and Stefano Fenu. My CS 2316 Head TA, Kate Unsworth also took on additional tasks during a semester in which I was the new CS 2316 instructor following Jay Summet's departure. Joshua Diaddigo and Keith Cartledge wrote the web site and IntelliJ plug-in used in my programmer study and joined me in several late-night hacking sessions to iron out the kinks and dry-run the programmer study. Bob Waters covered several CS 2340 lectures at crucial times that allowed me to make progress writing. Without their help I could not have finished.

My committee has been outstanding. Spencer guided me with great care through the software engineering portions of my thesis. Doug helped me learn agent-based simulation and apply an early version of AFABL to behavior modeling. An expert in AI for games and interactive narratives, Mark helped me place AFABL in context. Andrea Thomaz encouraged me to conduct the programmer study, which distinguishes my work from other work in reinforcement learning in programming languages.

\newpage

I am especially grateful to my advisor, Charles Isbell, for supporting me through years of personal struggle. Everyone knows that Charles is a brilliant researcher and leader. I know first hand that he is also an outstanding human being. No matter how difficult the challenges or how likely it seemed that I would fail, Charles always told me he believed I could finish. I am happy to have proved him right. I could not have finished without Charles's unwavering support and encouragement.

I met John Cortese when he gave a talk on quantum computing as part of his interview with GTRI in 2004. I asked a question he thought was insightful and after he was hired he sought me out to discuss it further. Over the ensuing years he became my best friend. John is not only one of the smartest people on earth (he's the only person in history to earn two Ph.D.s from Caltech, one in Electrical Engineering and one in Theoretical Physics), he has a gift for explaining advanced concepts to lay people, and he is one of the best human beings I know. Two years ago I was in despair. It seemed that the cumulative effects of family challenges and full-time work would prevent me from ever finishing my Ph.D. Without my reporting any of this to John -- as if he had some sort of sixth sense -- he called me out of the blue and offered to help me pay my bills so I could quit my job and finish my Ph.D. The next semester I went part-time and redoubled my efforts. Although I later went full-time again, John's support had reinvigorated me. It is no exaggeration to say that I owe this Ph.D. to John.

Finally I want to thank my wife, Caroline, who met me before I passed my qualifier, and after I became a full-time single father. She has endured years of being a ``Ph.D. widow'' while taking on the role of stepmother. Caroline has helped me in a million small and not so small ways as I struggled to manage a young family in distress. No matter how discouraged I became, no matter how sullen my mood, every time our eyes met she smiled. It is difficult to overstate the impact of such a seemingly small gesture. I would not be here without Caroline.

\clearpage
%\pagenumbering{gobble}  %remove page number on summary page
