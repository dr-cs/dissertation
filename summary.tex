\clearpage
\begin{centering}
\textbf{SUMMARY}\\
\vspace{\baselineskip}
\end{centering}

Improving the composability of modules in modular reinforcement learning and integrating modular reinforcement learning into a programming language supports reuse in adaptive agent software engineering.  This thesis claims that (1) modular reinforcement learning can be extended to support composability by decoupling the reward scales of the modules that comprise the agents, and (2) integrating modular reinforcement learning into a programming language supports adaptive agent programming by reducing the effort required to write agent programs and adapt them to new domains.

Composability, an essential property of modularity in software engineering, allows components to be reused in new systems.  In the case of a modular reinforcement learning agent composability means being able to reuse behavior modules in new agents without modifying the modules.  The current state of the art in modular reinforcement learning supports decomposition but not composition, or module reuse.  This dissertation contributes a reformulation of modular reinforcement learning -- command arbitration -- that supports composition by uncoupling the reward scales of the modules that comprise a modular reinforcement learning agent, and a command arbitration algorithm for modular reinforcement learning.

An excellent way to support software engineering is with practical, usable programming languages.  The second major contribution of this dissertation is a domain-specific language and framework embedded in the Scala programming language that integrates modular reinforcement learning.  We show how this integration is useful to software engineers writing practical adaptive agent software by applying AFABL to representative agent programming tasks and measuring the benefit of integrated modular reinforcement learning compared to traditional programming.  This application to practical software engineering problems distinguishes AFABL from previous work in integrating RL into programming languages such as ALisp.

%\pagenumbering{gobble}  %remove page number on summary page
